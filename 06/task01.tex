In the lecture, the force-length relation of a 1D polymer was calculated
using the microcanonical ensemble.

\paragraph{a) Now do the calculation in the canonical ensemble.
    (\textit{2 points}) \\
    HINT: Consider again a chain of $N$ monomers of size $a$ with $x_i=+a$ or 
    $-a$ and $X=\sum_ix_i$ the total length of the polymer (for the given 
    configuration). Note that the polymer is "ideal", i.e. the distribution of 
    bonds does not change the energy of the chain in the force-free case;
    in turn, under an external force $F$ the energy of this system is given by 
    $E=-FX$.
} \ \\
    \\
    % contour length L_0=Na
    % effective/projected length L (here: called X)
    Number of different states for given $N_+$ and $N+-$:
    \begin{equation}
        \Omega=\frac{N!}{N_+!N_-!}
    \end{equation}
    Partition sum:
    \begin{equation}
        Z=\Omega\cdot\sum_{X=0}^{Na}\exp(-\beta E_X)
        \ \ \ \ \ \ \ \textnormal{with }
        E_X=-FX
    \end{equation}
    Thermodynamical free energy:
    \begin{equation}
        F=k_BT\cdot\log(Z)
    \end{equation}
    \begin{itemize}
        \item get equations of state
        \item differentiate free energy for ?
        \item $\Rightarrow$ relationship between force and length $X$
    \end{itemize}
    ...
    \begin{equation}
        F=k_BT\cdot\frac{L}{Na^2} \textcolor{red}{\ \ \ \ (?)}
    \end{equation}

\paragraph{b) Show that your result is identical to the one in the lecture.
    Discuss the limits $Fa<<k_BT$ and $Fa>>k_BT$. (\textit{2 points})
} \ \\
    \\
    ...

