Consider the Debye model for the specific heat of a crystal as discussed 
in the lecture.

\paragraph{a) \textit{Corrections to classical limit:} In the lecture it was 
    shown that the Debye model yields $E=3Nk_BT$ in the classical limit as 
    expected. Calculate the first two corrections from a Taylor expansion in 
    $T_D/T$ of the integrand of the energy formula and discuss the 
    resulting heat capacity. (\textit{4 points})
} \ \\
    \\
    The specific heat is given by
    \begin{equation}
        c_v(T)
        =\frac{9Nk_B}{u^3_m}\cdot\int_0^{u_m}
        \frac{e^u\cdot u^4}{(e^u-1)^2}\cdot\dd u
        \ \ \ \ \ \ \textnormal{with }
        u:=\beta\hbar\omega=\frac{\hbar\omega}{k_BT}
        =\frac{\hbar\omega}{k_BT_D}\cdot\frac{T_D}{T} \,,
    \end{equation}
    where $u_m = \beta\hbar\omega_D = T_D/T$ because $T_D = \hbar\omega_D/k_B$.
    This integral can not be solved analytically, which is why we approximate
    the integrand via Taylor expansion up to the second order for small $u$,
    i.e. for large $T$:
    \begin{equation}
        \frac{e^u\cdot u^4}{(e^u-1)^2}
        \approx u^2+\frac{u^4}{12}+\frac{u^6}{240} \,.
    \end{equation}
    Thus, the specific heat can be approximated by
    \begin{equation}
        c_v(T)
        \approx\frac{9Nk_B}{u^3_m}\cdot\int_0^{u_m}
        u^2+\frac{u^4}{12}+\frac{u^6}{240}\cdot\dd u
        = 3Nk_B\cdot\left[1+\frac{1}{20}\left(\frac{T_D}{T}\right)^2+\frac{1}{560}\left(\frac{T_D}{T}\right)^4\right] \,.
    \end{equation}
    With the formula $c_v(T) = \dd E/\dd T$, we find for the energy corrections
    \begin{equation}
        E
        = 3Nk_B\cdot\left[T-\frac{1}{20}\frac{T_D^2}{T}-\frac{1}{1680}\frac{T_D^4}{T^3}\right] \,.
    \end{equation}

\paragraph{b) \textit{Number of phonons:} Write the mean number of phonons 
    $\langle N\rangle$ in the Debye model as an integral (similar but 
    different to the integral formula for the energy) and evaluate it for low 
    and high temperatures. (\textit{4 points})
} \ \\
    \\
    In the script, we had the following formula for the mean energy:
    \begin{align}
        E
        =\sum_\textnormal{modes}\hbar\omega
        \bigg(\frac{1}{e^{\beta\hbar\omega-1}}+\frac{1}{2}\bigg)
        \overset{!}{=}\hbar\omega
        \bigg(N+\frac{1}{2}\bigg) \,.
    \end{align}
    We conclude that the formula for the mean particle number will be given by:
    \begin{align}
        N
        =\sum_\textnormal{modes}
        \frac{1}{e^{\beta\hbar\omega-1}}
        = 3N\int_0^{\omega_D} \dd\omega \ \frac{3\omega^2}{\omega_D^3}\frac{1}{e^{\beta\hbar\omega}-1}
    \end{align}  % see p.82 in script
    For low $T$ (large $\beta$):
    \begin{align}
        N
        &\approx 3N\int_0^{\omega_D} \dd\omega \ \frac{3\omega^2}{\omega_D^3}\frac{1}{e^{\beta\hbar\omega}} \ \longrightarrow 0 \,.
    \end{align}
    For large $T$ (small $\beta$):
    \begin{align}
        N
        &\approx 3N\int_0^{\omega_D} \dd\omega \ \frac{3\omega^2}{\omega_D^3}\frac{1}{\beta\hbar\omega} = \frac{9NT}{2T_D} \,.
    \end{align}
