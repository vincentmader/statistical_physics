As a classical model for paramagnetism one 
can consider a system of $N$ particles with 
the Hamiltonian
\begin{equation}
    \mathcal{H}=-hM
    \ \ \ \textnormal{with}\ \ \ 
    M=\mu\cdot\sum_{i=1}^N\cos\theta_i
\end{equation}
where $h$ is an external homogeneous magnetic 
field, $\mu$ the magnetic moment of a single 
particle and $\theta_i$ the angle between the
magnetic field $h$ and the magnetic moment
$\mu$ of particle $i$.

\paragraph{1. Use the canonical distribution 
    to calculate the average magnetization, 
    $\langle M\rangle$, as a function of $h$
    and temperature $T$. 
    \textit{(3 points)}
}

\paragraph{2. The ratio of which quantities 
    determines the average magnetization? 
    Sketch the functional dependence of the 
    average magnetization on this ratio. 
    \textit{(1 point)}
}

\paragraph{3. Discuss the two limiting cases: 
    high temperature/weak field vs. 
    low temperature/strong field.
    \textit{(2 points)}
}
