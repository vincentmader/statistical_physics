In the lecture we have derived the following 
RG flow equation for the coupling constant $K$ 
of the Ising chain without magnetic field: 
the new value $K'$ is given by
\begin{equation}
    K'(K)=\frac{1}{2}\ln\cosh(2K).
\end{equation}
In addition we have derived the absolute 
increase in free energy per spin arising in 
each iteration:
\begin{equation}
    g(K)=\frac{1}{2}\ln2
    +\frac{1}{4}\ln\cosh(2K)
\end{equation}

\paragraph{1. Write a short computer program 
    (e.g. in Mathematica or Python) that 
    defines the flow equation $K'(K)$ and the 
    free energy increase $g(K)$ as functions. 
    Start with a coupling constant $K_0=1$ and 
    iterate through $K_1$, $K_2$, $K_3$ up to 
    $K_4$. Also calculated the corresponding 
    values $g_0=g(K_0)$ to $g_4=g(K_4).$ What 
    are the limits for these two series? 
    \textit{(1.5 points)}
}

\paragraph{2. Use these results to estimate 
    the dimensionless free energy per spin 
    $f=-\beta F/N$ in fourth order (simply cut 
    the appropriate sum after the term with 
    $g_4$; you can also include the next order
    term, but now by simply using the first 
    term in $g(K)$). Compare to the known 
    exact result for the Ising chain. How good
    is the numerical agreement?
    \textit{(1.5 points)}
}
