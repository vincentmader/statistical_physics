Consider a one-dimensional lattice model for a 
non-ideal gas with $N$ lattice sites and 
periodic boundary conditions. Each lattice 
site $i$ is either empty (occupancy $n_i=0$) 
or occupied by at most one atom (occupancy 
$n_i=1$). There is an attractive energy $J$ 
between atoms occupying neighbouring sites.
The chemical potential of the atoms is $\mu$. 
The Hamiltonian of this lattice gas is
\begin{equation}
    H=-J\cdot\sum_{\langle ij\rangle}n_in_j
    -\mu\cdot\sum_in_i \,,
\end{equation}
where $\sum_{\langle ij\rangle}$ is the sum 
over all pairs of neighbouring sites.

\paragraph{1. Express the partition sum of the 
    one-dimensional lattice gas in terms of 
    the transfer matrix $T$.  Calculate the 
    transfer matrix $T$ and its eigenvalues. 
    \textit{(2 points)}
} \ \\
\\
The partition sum is given by
\begin{align}
	Z_N = \sum_{\{n_i\}} e^{-\beta H} = 
	\sum_{n_1=0,1} ... \sum_{n_N=0,1} e^{-\beta H} \,.
\end{align}
Let us rewrite the Hamiltonian in the following manner:
\begin{align}
	H &= - J \sum_{\langle ij\rangle} n_i n_j 
	- \mu \sum_{i=1}^{N} n_i \notag \\
	&= - J \sum_{i=1}^{N} n_i n_{i+1} - 
	\frac{1}{2} \mu \sum_{i=1}^{N} (n_i + n_{i+1}) \,.
\end{align}
This is equivalent since we have periodic boundary conditions 
($n_{N+1}=n_1$) and the lattice is one-dimensional. 
Now, we can define
\begin{align}
	-\beta H = K \sum_{i=1}^{N} n_i n_{i+1} 
	+ \frac{1}{2} L \sum_{i=1}^{N} (n_i + n_{i+1}) \,,
\end{align}
where $K = \beta J$ and $L = \beta \mu$. 
This allows us to write the argument of the partition sum as
\begin{align}
	e^{-\beta H} = T_{1,2} \cdot T_{2,3} \cdot ... \cdot  T_{N,1} \,,
\end{align}
where 
\begin{align}
	T_{i,i+1} = e^{K n_i n_{i+1} + \frac{1}{2} L (n_i + n_{i+1})}
\end{align}
is the transfer matrix from lattice site $i$ to site $i+1$. 
From the fact that there are two possible values for $n_i$ 
and $n_{i+1}$ (unoccupied or occupied) which are independent 
of the index $i$, one can conclude that the $2\times 2$ transfer 
matrix is independent of the index $i$. Thus, one can write
\begin{align}
	T =
	\begin{bmatrix}
    1                & e^{\frac{1}{2}L} \\
    e^{\frac{1}{2}L} & e^{K + L}
    \end{bmatrix} \,,
\end{align}
where $T_{00}$ corresponds to $n_i=n_{i+1}=0$, $T_{01}$ corresponds 
to $n_i=0$ and $n_{i+1}=1$, $T_{10}$ corresponds to $n_i=1$ and 
$n_{i+1}=0$ and $T_{11}$ corresponds to $n_i=n_{i+1}=1$.
Since the transfer matrix is independent of the index $i$, on 
can simplify the partition sum further by defining
\begin{align}
	\vert n_i = 0 \rangle &= 
	\begin{bmatrix}
    1 \\
    0
    \end{bmatrix} \,, \\
    \vert n_i = 1 \rangle &= 
	\begin{bmatrix}
    0 \\
    1
    \end{bmatrix} \,.
\end{align}
With this definition it follows that 
$\sum_{n_i=0,1} \vert n_i\rangle \langle n_i\vert = 1$ and 
\begin{align}
	Z_N &= \sum_{n_1=0,1} ... \sum_{n_N=0,1}
	T_{1,2} \cdot T_{2,3} \cdot ... \cdot T_{N,1} \\
	&= \sum_{n_1=0,1} ... \sum_{n_N=0,1}
	\langle n_1 \vert T \vert n_2 \rangle 
	\langle n_2 \vert T \vert n_3 \rangle ... 
	\langle n_N \vert T \vert n_1 \rangle \\
	&= \sum_{n_1=0,1} 
	\langle n_1 \vert T^N \vert n_1 \rangle \\
	&= \lambda_1^N(K,L) + \lambda_2^N(K,L) \,,
\end{align}
where $\lambda_1$ and $\lambda_2$ are the eigenvalues of $T$ 
which are given by
\begin{align}
	\lambda_{1/2}(K,L) = 
	\frac{1+e^{K+L}\pm\sqrt{(1+e^{K+L})^2-4e^L(e^K-1)}}{2} \,. 
\end{align}

\paragraph{2. Find a transformation of the 
    occupancies $n_i$ to map the lattice gas 
    model to the Ising model with spins $s_i$. 
    \textit{(2 points)}
} \ \\
\\
A map from the lattice gas model,
\begin{align}
	\label{lattice}
	Z_L(K,L) = \sum_{\{n_i\}} 
	e^{K\sum_i n_i n_{i+1}+L \sum_i n_i} \,,
\end{align}
with occupancies $n_i=0,1$ to the Ising model,
\begin{align}
	\label{ising}
	Z_I(K',L') = \sum_{\{s_i\}} 
	e^{K'\sum_i s_i s_{i+1}+L' \sum_i s_i} \,,
\end{align}
with spins $s_i=\pm 1$ can be given by
\begin{align}
	\label{map}
	n_i = \frac{s_i + 1}{2} \,.
\end{align}
Inserting Eq.~\eqref{map} into Eq.~\eqref{lattice} yields
\begin{align}
	Z_L(K,L) = e^{\frac{N}{2}(\frac{K}{2} + L)} \cdot
	Z_I\left(K'(K,L),L'(K,L)\right) \,,
\end{align}
with $K'(K,L) = K/4$ and $L'(K,L)=(K+L)/2$. Thus, 
the lattice gas model and the Ising model are in fact 
equivalent. Expressing the eigenvalues of the lattice 
model $\lambda_{1/2}(K,L)$ with the new variables 
$K'$ and $L'$, 
\begin{align}
	\lambda_{1/2}\left(K',L'\right) = 
	e^{L'}\left(\cosh(L')\pm\sqrt{\cosh^2(L')-2e^{-2K'\sinh(2K')}}\right) \,,
\end{align}
and calculating the partition sum of the Ising model,
\begin{align}
	Z_I\left(K',L'\right) &= e^{-N(L'-K')} \, Z_L(K',L') 
	= e^{-N(L'-K')} \left(\lambda_1^N\left(K',L'\right)+\lambda_2^N\left(K',L'\right)\right) \notag \\
	&= \tilde{\lambda}_1^N\left(K',L'\right) + \tilde{\lambda}_2^N\left(K',L'\right) \,,
\end{align}
gives in fact the correct eigenvalues of the Ising model:
\begin{align}
	\tilde{\lambda}_{1/2}\left(K',L'\right) = 
	e^{K'}\left(\cosh(L')\pm\sqrt{\cosh^2(L')-2e^{-2K'\sinh(2K')}}\right) \,.
\end{align}


\paragraph{3. Derive an expression for the 
    average $\langle n_i\rangle$ in the limit 
    of $N\to\infty$ in terms of the 
    eigenvalues of the transfer matrix.
    \textit{(2 points)}
} \ \\
\\
Since $\lambda_1>\lambda_2$ the partition sum can be simplified for $N\to\infty$:
\begin{align}
	Z = \lambda_1^N + \lambda_2^N
	= \lambda_1^N \left[1+\left(\frac{\lambda_2}{\lambda_1}\right)^N\right] \,
	\longrightarrow \, \lambda_1^N \,.
\end{align}
We see that only $\lambda_1$ matters in the limit $N\to\infty$. Thus, we will 
neglect $\lambda_2$ in the following. \\
An expression for $\langle n_i\rangle$ can be given as:
\begin{align}
	\langle n_i\rangle &= \frac{1}{Z} \sum_{n_1=0,1}...\sum_{n_N=0,1}
	n_i \, e^{-\beta H} \notag \\
	&= \frac{1}{Z} \sum_{n_1=0,1} ... \sum_{n_N=0,1} n_i \,
	T_{1,2} \cdot T_{2,3} \cdot ... \cdot T_{N,1} \notag \\
	&= \frac{1}{Z} \sum_{n_1=0,1} ... \sum_{n_N=0,1}
	\langle n_1 \vert T \vert n_2 \rangle \,
	... \, n_i \, \langle n_i \vert T \vert n_{i+1} \rangle \, ... \,
	\langle n_N \vert T \vert n_1 \rangle \notag \\
	&= \frac{1}{Z} \sum_{n_1=0,1}
	\langle n_1 \vert T^{i-1} \vert n_i=1 \rangle
	\langle n_i=1 \vert T^{N-(i-1)} \vert n_1 \rangle \notag \\
	&\approx \frac{1}{Z} \, \lambda_1^N \,
	\langle n'_1=0 \vert n_i=1 \rangle
	\langle n_i=1 \vert n'_1=0 \rangle \notag \\
	&\approx \langle n'_1=0 \vert n_i=1 \rangle
	\langle n_i=1 \vert n'_1=0 \rangle \,.
\end{align}
Expressing the states in the eigenvector basis of $T$ and calculating 
the scalar products yields the wanted expression for $\langle n_i\rangle$ 
in the limit of $N\to\infty$. We know that $\vert n'_1=0 \rangle$ and 
$\vert n'_1=1 \rangle$ are the eigenvectors of $T$ with eigenvalues 
$\lambda_1$ and $\lambda_2$ respectively:
\begin{align}
	\vert n'_1=0 \rangle = 
	\begin{bmatrix}
    -e^{-\frac{L}{2}}\left(e^{K+L}-\lambda_1\right) \\
    1
    \end{bmatrix} \,, \quad
    \vert n'_1=1 \rangle = 
	\begin{bmatrix}
    -e^{-\frac{L}{2}}\left(e^{K+L}-\lambda_2\right) \\
    1
    \end{bmatrix} \,,
\end{align}
and that they form a basis. Thus, we have
\begin{align}
	\langle n'_1=0 \vert n'_1=1 \rangle =
	1 + e^{-L}(e^{K+L}-\lambda_1)(e^{K+L}-\lambda_2) \,.
\end{align}
And finally:
\begin{align}
	\langle n_i\rangle = 
	\left( 1 + e^{-L}(e^{K+L}-\lambda_1)(e^{K+L}-\lambda_2) \right)^2 \,.
\end{align}