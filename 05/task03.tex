Assume that we know the fundamental equation as 
\begin{equation}
    E=E(S,V,N) 
\end{equation}
for some well-behaved system of interest (i.e. at least $C^2$). Then it should 
not matter in which sequence we take the partial derivatives for a second 
derivative like $\frac{\partial^2 E}{\partial S\partial V}$.

\subsection{Derive the Maxwell relation}
We introduce the notation $\partial_x:=\frac{\partial}{\partial x}$
\begin{align}
    \frac{\partial^2 E}{\partial S\partial V}
    &=\frac{\partial^2 E}{\partial V\partial S} \\
    \partial_S\partial_V E &=\partial_V\partial_S E \\
    % \frac{\partial}{\partial S}\frac{\partial E}{\partial V}
    % &=
    % \frac{\partial}{\partial V}\frac{\partial E}{\partial S} \\
    -\partial_S p\ \bigg|_{V,N} &= \partial_V T\ \bigg|_{S,N} 
\end{align}
For this, we used the fact that
\begin{equation}
    \partial_S E\ \bigg|_{V,N}=T
    \ \ \ \ \ \ \textnormal{and}\ \ \ \ \ \
    \partial_V E\ \bigg|_{S,N}=-p
\end{equation}
which can easiliy be verified using the fundamental equation:
\begin{equation}
    dE=T\cdot dS-p\cdot dV+\mu\cdot dN
\end{equation}
There are three Maxwell relations, one for each of the energy terms.


\subsection{Confirm Maxwell relation explicitly for the ideal gas}
