Enclosed in a box of volume $V$ are $N_1$ molecules of species 1 with mass $m_1$
and $N_2$ molecules of species 2 with mass $m_2$. The system can be 
considered as an ideal gas with the total energy $E$.

\subsection{Phase space volume}
For a single species of an ideal gas:
\begin{equation}
    \Omega(E,\delta E)=\frac{1}{h^{3N}N!}\cdot
    \int_{E-\delta E\leq H(\vec q,\vec p)\leq E}
    d\vec q d\vec p
\end{equation}
For two different species with $N_1>>1$ and $N_2>>1$, 
the phase space volume $\Omega(E,V,N_1,N_2)$ ... \\
\\
\textcolor{red}{...}

\subsection{Entropy as a function of the ratio $N_1/N_2$}

\subsection{Total and partial pressures}
Total pressure:
\begin{equation}
    p=T\bigg(\frac{\partial S}{\partial V}\bigg)_{E,N_1,N_2}
\end{equation}
Partial pressures:
\begin{equation}
    p_i=...
\end{equation}

\subsection{Entropy of mixing}
Consider a situation in which the container is separated into two volumes $V_1$
and $V_2$ containing only molecules of species 1 and 2, respectively. The 
total energy of the system is $E$ and the two compartments are in thermal and 
mechanical equilibrium. 

\paragraph{Entropy of the demixed state} \ \\
From the lecture:
\begin{equation}
    \Omega_i(E)
    =\frac{V^{N_i}\cdot\pi^{3N_i/2}\cdot(2mE)^{3N_i/2}}
    {h^{3N_i}\cdot N_i!\cdot(\frac{3N_i}{2}!)}
\end{equation}
The total entropy is the sum of the individual entropies:
\begin{align}
    S&=S_1+S_2 \\
     &=k_B\cdot\bigg(\log(\Omega_1)+\log(\Omega_2)\bigg) \\
     &=k_B\cdot\log(\Omega_1\cdot\Omega_2)
\end{align}

\paragraph{Entropy of the mixture} \ \\
\\
\textcolor{red}{comparison}
