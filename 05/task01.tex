Enclosed in a box of volume $V$ are $N_1$ molecules of species 1 with mass $m_1$
and $N_2$ molecules of species 2 with mass $m_2$. The system can be 
considered as an ideal gas with the total energy $E$.

\subsection{Phase space volume}
    For a single species of an ideal gas:
    \begin{equation}
        \Omega(E)=\frac{1}{h^{3N}N!}\cdot
        \int \prod_{i=1}^{3N} dq_i \int \prod_{i=1}^{3N} dp_i \ \delta\left(E - H(q_i, p_i)\right)
    \end{equation}
    % h^3 is minimum phase phase volume
    % N! because of indistinguishable particles/states
    % integrate over all particles' positions and momenta for which H < E
    For two different species with $N_1>>1$ and $N_2>>1$, 
    the phase space volume $\Omega(E,V,N_1,N_2)$ is
    \begin{equation}
        \Omega(E)=\frac{V^N}{h^{3N}N_1!N_2!}\cdot
        \int \prod_{i=1}^{3N_1} dp_i \int \prod_{j=3N_1+1}^{3N} dp_j \ \delta\left(E - H(q_i, p_i)\right)
    \end{equation}
    The factorials arise from the fact that we are dealing with two independent gas species and the integral over $dq$
    is carried out already and gives the volume $V^N$ since the Hamiltonian 
    \begin{equation}
    H(q_i, p_i) = \sum_{i=1}^{3N_1}\frac{p_i^2}{2m_1} + \sum_{j=3N_1+1}^{3N}\frac{p_j^2}{2m_2}
	\end{equation}        
    is independent of the position.
    At this point we can perform the momentum transformation
    \begin{align}
        \bar{\mathsf{p}}_i &= \frac{p_i}{\sqrt{2m_1}} \\
        \bar{\mathsf{p}}_j &= \frac{p_j}{\sqrt{2m_2}}
    \end{align}
    to give 
    \begin{align}
        \Omega(E)&=\frac{V^N (2m_1)^{\frac{3N_1}{2}} (2m_2)^{\frac{3N_2}{2}}}{h^{3N}N_1!N_2!}\cdot
        \int \prod_{i=1}^{3N} d\bar{\mathsf{p}}_i \ \delta\left(E - \sum_{i=1}^{3N}\bar{\mathsf{p}}_i^2\right) \\
        &=\frac{V^N (2m_1)^{\frac{3N_1}{2}} (2m_2)^{\frac{3N_2}{2}}}{h^{3N}N_1!N_2!}\cdot
        \frac{\pi^{\frac{3N}{2}} E^{\frac{3N}{2}}}{(\frac{3N}{2})!}
    \end{align}
    which was again just an integration over a $3N$-dimensional sphere of radius $\sqrt{E}$ (which is a good approximation for large $N$). \\
    Introducing the reduced mass $\mu = m_1 m_2 / (m_1 + m_2)$ one can find the following result
    \begin{align}
        \Omega(E)=\frac{V^N \pi^{\frac{3N}{2}} (2\mu E)^{\frac{3N}{2}}}{h^{3N}N!\left(\frac{3N}{2}\right)!}\cdot
        \frac{N!}{N_1!N_2!}\left( \frac{m_1}{\mu} \right)^{\frac{3N_1}{2}}\left( \frac{m_2}{\mu} \right)^{\frac{3N_2}{2}} \,.
    \end{align}
    

\newpage
\subsection{Entropy as a function of the ratio $N_1/N_2$}
    The entropy of the system will be given by
    \begin{align}
        S &= k_B \ln \Omega(E) \\
        &= k_B \left[\ln\left( \frac{V^N \pi^{\frac{3N}{2}} (2\mu E)^{\frac{3N}{2}}}{h^{3N}N!\left(\frac{3N}{2}\right)!} \right)
         + \ln\left( \frac{N!}{N_1!N_2!}\left( \frac{m_1}{\mu} \right)^{\frac{3N_1}{2}}\left( \frac{m_2}{\mu} \right)^{\frac{3N_2}{2}} \right)\right]
    \end{align}
    The first term corresponds to the standard result for an ideal gas of $N$ identical particles with mass $\mu$
    \begin{equation}
    S_{\text{id}} = k_B N \left\{ \ln\left[ \frac{V}{N} \left( \frac{4\pi\mu E}{3h^2N} \right)^{\frac{3}{2}}  \right] + \frac{5}{2} \right\} 
    \end{equation}
    and the second term is given by
    \begin{equation}
    S_{\text{mixed}} = k_B \left[ N\ln N - N_1\ln N_1 - N_2\ln N_2 + N_1 \ln\left(\frac{m_1}{\mu}\right)^{\frac{3}{2}} + N_2 \ln\left(\frac{m_2}{\mu}\right)^{\frac{3}{2}} \right]
    \end{equation}
    using Stirling's approximation formula. Defining the ratio $\eta = N_1/N_2$, the mixed term can be rewritten to
    \begin{equation}
    S_{\text{mixed}} = k_B N \left[ \frac{\eta}{1+\eta}\left( \ln\left( \frac{m_1}{\mu} \right)^{\frac{3}{2}} - \ln\frac{\eta}{1+\eta}\right) 
    + \frac{1}{1+\eta}\left( \ln\left( \frac{m_2}{\mu} \right)^{\frac{3}{2}} - \ln\frac{1}{1+\eta}\right) \right] \,.
    \end{equation}
    To find the maximum entropy, we need only differentiate the mixed contribution since the identical contribution is not a function of $\eta$
    \begin{equation}
    \frac{\partial S}{\partial \eta} = 0 \, \, \,  \longrightarrow  \, \, \,  \eta_0 = \frac{N_1}{N_2} = \left(\frac{m_1}{m_2}\right)^{\frac{3}{2}} \,.
    \end{equation}
    
    
\subsection{Total and partial pressures}
    To calculate the pressure, using the equation given, we need only differentiate $S_{\text{id}}$ since $S_{\text{mixed}}$ is not a function of volume
    \begin{equation}
        p=T\bigg(\frac{\partial S_{\text{id}}}{\partial V}\bigg)_{E,N_1,N_2} = \frac{N k_B T}{V} = \frac{N_1 k_B T}{V} + \frac{N_2 k_B T}{V} = p_1 + p_2
    \end{equation}
    So we can see both gases contribute to the pressure according to their number of particles.

\newpage
\subsection{Entropy of mixing}
    Consider a situation in which the container is separated into two volumes 
    $V_1$ and $V_2$ containing only molecules of species 1 and 2, respectively. 
    The total energy of the system is $E$ and the two compartments are in 
    thermal and mechanical equilibrium. \\
    \\
    From the lecture, we know the entropy for the unmixed gas:
    \begin{equation}
        S = S_1 + S_2
    \end{equation}
    with 
    \begin{align}
    S_1 &= k_B N_1 \left\{ \ln\left[ \frac{V_1}{N} \left( \frac{4\pi m_1 E}{3h^2N} \right)^{\frac{3}{2}}  \right] + \frac{5}{2} \right\} \,, \\
    S_2 &= k_B N_2 \left\{ \ln\left[ \frac{V_2}{N} \left( \frac{4\pi m_2 E}{3h^2N} \right)^{\frac{3}{2}}  \right] + \frac{5}{2} \right\} \,.
    \end{align}
    Rewriting it with the reduced mass $\mu$ yields
    \begin{align}
    S_1 &= k_B N_1 \left\{ \ln\left[ \frac{V}{N} \left( \frac{4\pi \mu E}{3h^2N} \right)^{\frac{3}{2}}  \right] + \frac{5}{2} + \ln\left( \frac{m_1}{\mu} \right)^{\frac{3}{2}} \right\} \,, \\
    S_2 &= k_B N_2 \left\{ \ln\left[ \frac{V}{N} \left( \frac{4\pi \mu E}{3h^2N} \right)^{\frac{3}{2}}  \right] + \frac{5}{2} + \ln\left( \frac{m_2}{\mu} \right)^{\frac{3}{2}} \right\} \,.
    \end{align}
    The entropy after mixing will be higher than before, the change in entropy will be
    \begin{align}
    \Delta S = S_{\text{id}} + S_{\text{mixed}} - S_1 - S_2 = - k_B \left( N_1 \ln\frac{N_1}{N} + N_2 \ln\frac{N_2}{N} \right) \,.
    \end{align}
