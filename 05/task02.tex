Schwarzschild radius:
\begin{equation}
    R=\frac{2GM}{c^2}
    \label{eq:schwarzschild}
\end{equation}
Hawking radiation:
\begin{equation}
    T=\frac{\hbar c^3}{8\pi GMk_B}
    \label{eq:hawking}
\end{equation}
Einstein's mass-energy relation:
\begin{equation}
    E=Mc^2
    \label{eq:einstein}
\end{equation}

\subsection{Specific heat $c_V$}
From \autoref{eq:hawking}:
\begin{equation}
    \Rightarrow M=\frac{\hbar c^3}{8\pi Gk_BT}
\end{equation}
Assumption: inner energy $U=E$. \\
\\
With \autoref{eq:einstein} and $c_V=\frac{\partial U}{\partial T}$ follows:
\begin{align}
    E&=\frac{\hbar c^5}{8\pi Gk_BT} \\
    \Rightarrow c_V&=\frac{\partial E}{\partial T}
    =-\frac{\hbar c^5}{8\pi Gk_BT^2}
\end{align}

\subsection{Entropy $S$}
To get an expression for the entropy, we first plug 
\autoref{eq:einstein} in \autoref{eq:hawking}:
\begin{equation}
    T(E)=\frac{\hbar c^5}{8\pi Gk_BE}
\end{equation}
Next we use the definition of temperature:
% from minium entropy principle
\begin{equation}
    \frac{\partial S(E)}{\partial E}:=\frac{1}{T(E)}
    =\frac{8\pi Gk_BE}{\hbar c^5}
\end{equation}
Integration leads to
\begin{equation}
    S(E)=\frac{4\pi Gk_BE^2}{\hbar c^5}
\end{equation}
The integration constant is set to zero to be consistent with 
the condition $S(M=0)\overset{!}{=}0$. \\
\\
Plugging first \autoref{eq:einstein} and then \autoref{eq:schwarzschild} into 
this result leads to
\begin{align}
    S
    % &=\frac{4\pi Gk_BE^2}{\hbar c^5} \\
    % &=\frac{\hbar G}{c^3}\frac{4\pi k_BE^2}{\hbar^2 c^2} \\
    % &=L_P\frac{4\pi k_BE^2}{\hbar^2c^2}
    % \frac{\sqrt{\hbar G/c^3}}{\sqrt{\hbar G/c^3}}
    &=\frac{4\pi Gk_BM^2c^4}{\hbar c^5} \\
    &=\frac{\pi k_BR^2c}{G\hbar} \\
    &=\frac{k_B}{4c^2}\frac{4\pi R^2}{L_P^2}
\end{align}
\textcolor{red}{$c^2$?} \\
As we can see, the entropy is directly proportional the surface area of a 
black hole measured in units of the Planck length $L_P=\sqrt{\hbar G/c^3}$.

\subsection{Information storage}
Thermodynamical entropy:
\begin{equation}
    S_\textnormal{t}=k_B\log{\Omega}
    \label{}
\end{equation}
Information entropy ($=$ number of bits):
\begin{equation}
    S_\textnormal{i}=\log_2(\Omega)
\end{equation}
Combining the last two relations yields
\begin{equation}
    S_\textnormal{i}=\log_2\bigg[\exp\bigg(\frac{S_t}{k_B}\bigg)\bigg]
\end{equation}
\textcolor{red}{
    TODO: calculate bit number for small black hole ($\sim1$cm) with $S_t(E(R))$
}

% dS = dQ / T
% S = k_B log(Omega)
