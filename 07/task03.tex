Consider a gas in contact with a solid surface. The molecules of the gas 
can adsorb to specific sites on the surface. These sites are sparsely 
enough distributed over the surface that they do not directly interact. 
In total, there are $N$ adsorption sites, and each can adsorb $n=0$, 
$n=1$ or $n=2$ molecules.  When an adsorption site is unoccupied, the 
energy of the site is zero. When an adsorption site is occupied by a 
single molecule, the energy of the site is $\varepsilon_1$. When an 
adsorption site is doubly occupied, the adsorption energy is 
$\varepsilon_2$. In addition, the two adsorbed molecules can interact in 
a vibrational mode with frequency $\omega$, so that the energy of the 
doubly occupied adsorption site is
$$\varepsilon_2+\nu\hbar\omega
\ \ \ \ \ \textnormal{with}\ \ \ \ \ 
\nu=0,1,2,...$$
The gas above the surface can be considered as a heat and particle 
reservoir with temperature $T$ and chemical potential $\mu$.

\paragraph{a) Calculate the grand canonical partition sum $Z_G$. 
    \textit{(2 points)}
} \ \\
\\

\paragraph{b) Calculate the grand canonical potential $\Psi$. 
    \textit{(1 point)}
} \ \\
\\

\paragraph{c) Calculate the mean number of adsorbed molecules on the 
    surface from $Z_G$. \textit{(1 point)}
} \ \\
\\

\paragraph{d) Calculate the mean number of adsorbed molecules on the 
    surface directly from $\Psi$ and convince yourself that it gives 
    the same as calculated in c). 
    \textit{(2 points)}
} \ \\
\\

\paragraph{e) Give the probability that an adsorption site is in the 
    state with $n=2$ and $\nu=3$. \textit{(1 point)}
} \ \\
\\
