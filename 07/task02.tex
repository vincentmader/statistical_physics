Consider a molecule, such as carbon monoxide, which consists of two 
different atoms, one carbon and one oxygen, separated by a distance $d$. 
Such a molecule can exist in quantum states of different orbital angular 
momentum. Each state has the energy
\begin{equation}
    \varepsilon_l=\frac{\hbar^2}{2I}\cdot l(l+1) \,,
\end{equation}
where $I=\mu d^2$ is the moment of inertia of the molecule about an axis 
through its centre of mass and $\mu$ is the reduced mass defined by 
\begin{equation}
    \frac{1}{\mu}=\frac{1}{m_1}+\frac{1}{m_2} \,.
\end{equation}
$l=0,1,2,...$ is the quantum number associated with the orbital angular 
momentum. Each energy level of the rotating molecule has the degeneracy
\begin{equation}
    g_l=2l+1 \,.
\end{equation}

\paragraph{a) Find the general expression for the canonical partition 
    function $Z$. \textit{(1 point)}
} \ \\
\\

\paragraph{b) Show that for high temperatures, $Z$ can be approximated 
    by an integral. Calculate the integral. HINT: For high T, find an 
    approximate integral representation for the summands at given $l$ and
    demonstrate that the integral can be extended over the complete 
    summation range. \textit{(2 points)}
} \ \\
\\

\paragraph{c) Evaluate the high temperature mean energy $E$ and the heat 
    capacity $C_V$. \textit{(2 points)}
} \ \\
\\

\paragraph{d) Find the low-temperature approximations to the canonical 
    partition function, the mean energy $E$ and the heat capacity $C_V$.
    \textit{(3 points)}
} \ \\
\\
