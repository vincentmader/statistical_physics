Consider a molecule, such as carbon monoxide, which consists of two 
different atoms, one carbon and one oxygen, separated by a distance $d$. 
Such a molecule can exist in quantum states of different orbital angular 
momentum. Each state has the energy
\begin{equation}
    \varepsilon_l=\frac{\hbar^2}{2I}\cdot l(l+1) \,,
\end{equation}
where $I=\mu d^2$ is the moment of inertia of the molecule about an axis 
through its centre of mass and $\mu$ is the reduced mass defined by 
\begin{equation}
    \frac{1}{\mu}=\frac{1}{m_1}+\frac{1}{m_2} \,.
\end{equation}
$l=0,1,2,...$ is the quantum number associated with the orbital angular 
momentum. Each energy level of the rotating molecule has the degeneracy
\begin{equation}
    g_l=2l+1 \,.
\end{equation}

\paragraph{a) Find the general expression for the canonical partition 
    function $Z$. \textit{(1 point)}
} \ \\
\\
    Definition of partition sum:
    \begin{align}
        Z=\sum_i\exp(-\beta E_i)
    \end{align}
    Let us define $k:=\frac{\beta\hbar^2}{2I}$. \\
    \\
    In the case of 
    a rigid rotator with degenerate energy levels, the partition sum becomes
    \begin{align}
        Z
        &=\sum_{l=0}^\infty g_l\cdot\exp\bigg(
            -k\cdot l(l+1)
        \bigg) \\
        &=\sum_{l=0}^\infty(2l+1)\cdot\exp\bigg(
            -k\cdot l(l+1)
        \bigg)
    \end{align}

\paragraph{b) Show that for high temperatures, $Z$ can be approximated 
    by an integral. Calculate the integral. HINT: For high $T$, find an 
    approximate integral representation for the summands at given $l$ and
    demonstrate that the integral can be extended over the complete 
    summation range. \textit{(2 points)}
} \ \\
\\
    For high temperatures, the energy spacing between the 
    individual energy levels becomes small (when compared to 
    $k_BT$). Thus, we can replace the summation by the integral
    % For $x\to0$, $\exp(x)\approx1+x$. Thus, for $\beta\to\infty$:
    \begin{align}
        Z
        &=\int_0^\infty (2l+1)\cdot\exp\bigg(
            -k\cdot l(l+1)
        \bigg)\cdot dl \\
        &=\int_0^\infty \exp\bigg(
            -k\cdot l(l+1)
        \bigg)\cdot d(l(l+1)) \\
        &=\frac{1}{k}=\frac{2I}{\hbar^2\beta}
        % &\approx\sum_l(2l+1)\cdot\bigg(
        %     1-kl(l+1)
        % \bigg) \\
        % &=\sum_l\bigg(
        %     2l+1-2kl^2(l+1)-kl(l+1)
        % \bigg) \\
        % &=\sum_l\bigg(
        %     2l+1-2kl^3-2kl^2-kl^2-kl
        % \bigg) \\
        % &=\sum_l\bigg(
        %     -2kl^3-3kl^2+(2-k)l+1
        % \bigg) \\
        % &=\sum_l\int_0^l\bigg(
        %     -6kl'^2-6kl'+2-k
        % \bigg)\cdot dl' \\
        % &=\sum_l\int_0^l\bigg(
        %     -6k(l'^2+l')+2-k
        % \bigg)\cdot dl' \\
        % &=\red{...} \\
        % &=\int_0^l\sum_l\bigg(
        %     -6k(l'^2+l')+2-k
        % \bigg)\cdot dl' \\
        % &=\int_0^l\bigg[
        %     \sum_l\bigg(
        %         -6k(l'^2+l')
        %     \bigg)+(2-k)\cdot l
        % \bigg]\cdot dl' \\
        % &=\red{...}
    \end{align}

\paragraph{c) Evaluate the high temperature mean energy $E$ and the heat 
    capacity $C_V$. \textit{(2 points)}
} \ \\
\\
    The mean energy $E$ (i.e. short for $\langle E\rangle$) can be 
    calculated from the partition sum
    \begin{align}
        E
        &=-\pder{\ln Z}{\beta} \\
        &=-\pder{}{\beta}\ln\bigg(
            \frac{2I}{\hbar^2\beta}
        \bigg) \\
        &=\frac{1}{\beta}=k_BT
    \end{align}
    Heat capacity:
    \begin{align}
        C_V
        =\pder{E}{T}
        % &=2k_BT\cdot\pder{\ln Z}{T}
        % +k_BT^2\cdot\frac{\partial^2\ln Z}{\partial T^2} \\
        =k_B
    \end{align}

\paragraph{d) Find the low-temperature approximations to the canonical 
    partition function, the mean energy $E$ and the heat capacity $C_V$.
    \textit{(3 points)}
} \ \\
\\
    At low temperatures, the majority of the particles will 
    be occupying the ground state. We can therefore approximate
    the partition function simply by ignoring all terms after 
    the first two:
    \begin{align}
        Z
        &=\sum_{l=0}^{\infty}(2l+1)\cdot\exp\bigg(
            -l(l+1)\cdot\frac{\hbar^2\beta}{2I}
        \bigg) \\
        &=1+3\cdot\exp\bigg(
            -\frac{\beta\hbar^2}{I}
        \bigg)
    \end{align}
    The average energy is thus
    \begin{align}
        E
        &=-\pder{}{\beta}\ln Z \\
        &=-\pder{}{\beta}\ln\bigg[
            1+3\cdot\exp\bigg(
                -\frac{\beta\hbar^2}{I}
            \bigg)
        \bigg] \\
        &=\frac{3\hbar^2/I}{e^{\beta\hbar^2/I}+3}
    \end{align}
    And the heat capacity:
    \begin{align}
        C_V
        &=\pder{\beta}{T}\pder{}{\beta}E \\
        &=-\frac{1}{k_BT^2}\cdot\pder{}{\beta}\bigg(
            \frac{3\hbar^2/I}{e^{\beta\hbar^2/I}+3}
        \bigg) \\
        &=\frac{3\hbar^4}{k_BT^2I^2}\cdot
        \frac{e^{\hbar^2\beta/I}}{e^{\beta\hbar/I}+3} \\
        &\approx3k_B\bigg(
            \frac{\hbar^2}{Ik_BT}
        \bigg)^2\cdot\exp\bigg(
            -\frac{\hbar^2}{Ik_BT}
        \bigg)
    \end{align}
    
