Consider a NaCl crystal that has a diatomic basis and therefore not only 
acoustic but also optical phonons. In the spirit of the Debye model, the 
density of states in the phonon spectrum can be approximated as
\begin{equation}
    D(\omega)
    =D_\textnormal{D}(\omega)+\delta(\omega-2\omega_\textnormal{D})
    \ \ \ \ \ 
    \textnormal{with}
    \ \ \ \ \ 
    D_\textnormal{D}(\omega)=
    \begin{cases}
        3\frac{\omega^2}{\omega_\textnormal{D}^3} 
        & \textnormal{for } \omega\le\omega_\textnormal{D} \\
        0 & \textnormal{for } \omega>\omega_\textnormal{D}
    \end{cases}
\end{equation}
The first contribution describes the acoustic phonons as discussed in 
the lecture. The second contribution approximately describes the optical 
phonons. 

\paragraph{a) Calculate the specific heat of the crystal for high and 
    low temperatures. \textit{(5 points)}
} \ \\
\\
