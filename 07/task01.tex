Consider a NaCl crystal that has a diatomic basis and therefore not only 
acoustic but also optical phonons. In the spirit of the Debye model, the 
density of states in the phonon spectrum can be approximated as
\begin{equation}
    D(\omega)
    =D_\textnormal{D}(\omega)+\delta(\omega-2\omega_\textnormal{D})
    \ \ \ \ \ 
    \textnormal{with}
    \ \ \ \ \ 
    D_\textnormal{D}(\omega)=
    \begin{cases}
        3\frac{\omega^2}{\omega_\textnormal{D}^3} 
        & \textnormal{for } \omega\le\omega_\textnormal{D} \\
        0 & \textnormal{for } \omega>\omega_\textnormal{D}
    \end{cases}
\end{equation}
The first contribution describes the acoustic phonons as discussed in 
the lecture. The second contribution approximately describes the optical 
phonons. 

\paragraph{a) Calculate the specific heat of the crystal for high and 
    low temperatures. \textit{(5 points)}
} \ \\
\\
    According to the Debye model, the modes are counted in the following way:
    \begin{equation}
	\sum_{\text{modes}} (...)
        =3N \int_0^{\infty}\dd\omega\, D(\omega)\, (...)
    \end{equation}
    Thus, the energy is given by
    \begin{align}
        E
        &=\sum_{\text{modes}}\hbar\omega
        \left(\frac{1}{e^{\beta\hbar\omega}-1}-\frac{1}{2}\right) \\
	&=E_0+3N\int_0^{\infty} \dd\omega\, D(\omega)\, 
        \frac{\hbar\omega}{e^{\beta\hbar\omega}-1} \,.
    \end{align}
    The specific heat $c_v$ is given by
    \begin{align}
	c_v(T)=\frac{\partial E}{\partial T}
	&=\frac{3\hbar^2N}{k_B T^2} \int_0^{\infty} \dd\omega \, D(\omega) \, 
	    \frac{e^{\beta\hbar\omega} \, 
            \omega^2}{\left(e^{\beta\hbar\omega}-1\right)^2} \notag \\
	&= \frac{3\hbar^2N}{k_B T^2} \int_0^{\infty} \dd\omega \, 
	    \left[ D_\textnormal{D}(\omega)+
            \delta(\omega-2\omega_\textnormal{D}) \right] \, 
	    \frac{e^{\beta\hbar\omega} \, 
            \omega^2}{\left(e^{\beta\hbar\omega}-1\right)^2} \notag \\
	&= \frac{9Nk_B}{u_m^3} \int_0^{u_m} \dd u \, 
            \frac{e^u \, u^4}{\left(e^u-1\right)^2}+\frac{12\hbar^2N}{k_B T^2}  
	    \frac{e^{2\beta\hbar\omega_D} \,
            \omega_D^2}{\left(e^{2\beta\hbar\omega_D}-1\right)^2} \,,
    \end{align}
    where the first term is well known from the lecture and 
    $u_m = \beta k_B T_D$ with the Debye temperature $T_D$ and 
    $\omega_D = k_B T_D / \hbar$. The second term approximately describes the 
    optical phonons. \\
    \\
    \\
    \\
    High temperature limit: \\
    \\
    We know that the first term becomes $c_{v,\text{D}}(T)\approx 3Nk_B$. The 
    second term can be approximated by $c_{v,\delta}(T)\approx 3Nk_B$. Thus
    \begin{equation}
	c_{v}(T) \approx 6Nk_B \,.
    \end{equation}
    Low temperature limit: \\
    \\
    We know that the first term becomes 
    $c_{v,\text{D}}(T)\approx\frac{12}{5}\pi^4Nk_B (T/T_D)^3$. The second term 
    can be approximated by $c_{v,\delta}(T)\approx 0$. Thus
    \begin{equation}
	c_{v}(T)\approx\frac{12\pi^4}{5}N k_B\left(\frac{T}{T_D}\right)^3 \,.
    \end{equation}
