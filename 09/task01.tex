The classical limit of quantum fluids arises as the 
linear term in a Taylor expansion in fugacity 
$z=e^{\beta\mu}$ around $z=0$ (corresponding to $\mu=-\infty$). 
This also corresponds to an expansion in density $\rho=N/V$
(the classical gas is a dilute gas, because then wavefunctions 
do not overlap). The quadratic terms in these expansions 
correspond to the first quantum corrections to the classical 
limit. As always, all quantities of interest can be calculated 
from the grandcanonical potential, which for the ideal quantum
fluids reads
\begin{equation}
    \Psi(T,V,\mu)=\mp g_Sk_BT\frac{V}{h^3}
    \int\vec o=p\ln\bigg[
        1\pm ze^{-\beta\frac{\vec p^2}{2m}}
    \bigg]
\end{equation}
with the different signs corresponding to fermions and bosons, 
respectively. From thermodynamics, we also know $\Psi=-pV$.

\paragraph{1. Expand the integrand in the formula for the 
    grandcanonical potential $\Psi$ to second order in z and
    perform the two integrals. \textit{(2 points)}
} \ \\
    \\

\paragraph{2. Calculate the mean particle number 
    $N =-\partial_\mu\Psi$ in the same order. Invert this 
    relation to get $u=\rho\lambda^3/g_s$ as a function of $z$. 
    (2.5 points)
} \ \\
    \\

\paragraph{3. Combine your results for $\Psi$ and $u$ to obtain 
    the first two terms for pressure p in an expansion in 
    $\rho$ to second order. \textit{(2.5 points)}
} \ \\
    \\

\paragraph{4. Discuss your results for the classical limit and 
    the first quantum correction. Where does degeneracy $g_s$
    show up? What do the differences in sign mean?
    \textit{(1 point)}
} \ \\
    \\
