The classical limit of quantum fluids arises as the 
linear term in a Taylor expansion in fugacity 
$z=e^{\beta\mu}$ around $z=0$ (corresponding to $\mu=-\infty$). 
This also corresponds to an expansion in density $\rho=N/V$
(the classical gas is a dilute gas, because then wavefunctions 
do not overlap). The quadratic terms in these expansions 
correspond to the first quantum corrections to the classical 
limit. As always, all quantities of interest can be calculated 
from the grandcanonical potential, which for the ideal quantum
fluids reads
\begin{equation}
    \Psi(T,V,\mu)=\mp g_Sk_BT\frac{V}{h^3}\cdot
    \int\dd\vec p\cdot\ln\bigg[
        1\pm\ ze^{-\beta\frac{\vec p^2}{2m}}
    \bigg]
    \label{eq:ideal_quantum_fluid}
\end{equation}
with the different signs corresponding to fermions and bosons, 
respectively. From thermodynamics, we also know $\Psi=-pV$.

\paragraph{1. Expand the integrand in the formula for the 
    grandcanonical potential $\Psi$ to second order in z 
    and perform the two integrals. \textit{(2 points)}
} \ \\
    \\
    Taylor expansion of the natural logarithm:
    \begin{align}
        \ln(1\pm\ x)
        &=\sum_{n=1}^\infty(-1)^{n+1}\cdot\frac{x^n}{n}
        =\pm\ x-\frac{x^2}{2}+O(x^3) 
        % \ln(1-x)
        % &=-\sum_{n=1}^\infty\frac{x^n}{n} \\
        % &=-x-\frac{x^2}{2}+O(x^3)
    \end{align}
    We define $C=g_Sk_BT\frac{V}{h^3}$ and plug the 
    definition of the Taylor expansion into 
    \autoref{eq:ideal_quantum_fluid}.
    \begin{align}
        \Psi(T,V,\mu)
        &\approx\mp C\cdot
        \int\dd\vec p\cdot\bigg[
            \pm\ ze^{-\beta\frac{\vec p^2}{2m}}
            -\frac{z^2}{2}e^{-2\beta\frac{\vec p^2}{2m}}
        \bigg] \\
        &=\mp 4\pi C\cdot\bigg[
            \pm\ z\int p^2\dd p\cdot
                e^{-\beta\frac{p^2}{2m}}
            -\frac{z^2}{2}\cdot\int p^2\dd p\cdot
                e^{-\beta\frac{p^2}{m}}
        \bigg]
    \end{align}
    We now identify $\ p^2e^{-\beta\frac{p^2}{2m}}
    =-2m\cdot\partial_\beta e^{-\beta\frac{p^2}{2m}}\ \ $
    and $\ \ p^2e^{-\beta\frac{p^2}{m}}
    =-m\cdot\partial_\beta e^{-\beta\frac{p^2}{m}}$:
    \begin{align}
        \Psi(T,V,\mu)
        &=\mp 4\pi C\cdot\bigg[
            \mp2mz\int \dd p\cdot\partial_\beta
                e^{-\beta\frac{p^2}{2m}}
            +\frac{mz^2}{2}\cdot\int \dd p\cdot\partial_\beta
                e^{-\beta\frac{p^2}{m}}
        \bigg] \\
        &=\mp 4\pi\cdot Cm\cdot\bigg[
            \mp2z\cdot\partial_\beta\int \dd p\cdot
                e^{-\beta\frac{p^2}{2m}}
            +\frac{z^2}{2}\cdot\partial_\beta\int \dd p\cdot
                e^{-\beta\frac{p^2}{m}}
        \bigg]
    \end{align}
    Substituting $q=\sqrt{\frac{\beta}{2m}}\cdot p$
    and $r=\sqrt{\frac{\beta}{m}}\cdot p$
    \begin{align}
        \Psi(T,V,\mu)
        &=\mp 4\pi\cdot Cm\cdot\bigg[
            \mp2z\cdot\partial_\beta
                \sqrt{\frac{2m}{\beta}}
                \int e^{-q^2} \dd q
            +\frac{z^2}{2}\cdot\partial_\beta
                \sqrt{\frac{m}{\beta}}
                \int e^{-r^2} \dd r
        \bigg] \\
        &=\mp 4\pi\cdot Cm\cdot\sqrt{m\pi}\cdot\bigg[
            \mp2\sqrt{2}z\cdot\partial_\beta
                \frac{1}{\sqrt{\beta}}
            +\frac{z^2}{2}\cdot\partial_\beta
                \frac{1}{\sqrt{\beta}}
        \bigg] \\
        &=\pm\ C\cdot
            \bigg(\frac{\pi m}{\beta}\bigg)^{3/2}
            \cdot
        \bigg[
            \mp2\sqrt{2}z
            +\frac{z^2}{2}
        \bigg] \\
        &=\pm\ g_sV\cdot k_BT\cdot
            \bigg(\frac{\pi m}{\beta h^2}\bigg)^{3/2}
            \cdot
        \bigg[
            \mp2\sqrt{2}z
            +\frac{z^2}{2}
        \bigg] \\
        &=\pm\ g_sV\cdot k_BT\cdot
            \bigg(\frac{\pi m}{\beta h^2}\bigg)^{3/2}
            \cdot
        \bigg[
            \mp\sqrt{8}e^{\beta\mu}
            +\frac{1}{2}e^{2\beta\mu}
        \bigg]
    \end{align}

\newpage
\paragraph{2. Calculate the mean particle number 
    $N =-\partial_\mu\Psi$ in the same order. Invert this 
    relation to get $u=\rho\lambda^3/g_s$ as a function of $z$. 
    (2.5 points)
} \ \\
    \\
    % \begin{align}
        % \Psi(T,V,\mu)
        % &=\pm\ g_sV\cdot k_BT\cdot\bigg(
        %     \frac{\pi m}{\beta h^2}
        % \bigg)^{3/2}\cdot\bigg[
        %     \mp\sqrt{8}e^{\beta\mu}
        %     +\frac{1}{2}e^{2\beta\mu}
        % \bigg]
    % \end{align}
    Particle number $N=-\partial_\mu\Psi$:
    \begin{align}
        N
        &=\mp\ g_sV\cdot k_BT\cdot\beta\bigg(
            \frac{\pi m}{\beta h^2}
        \bigg)^{3/2}\cdot\bigg[
            \mp\sqrt{8}e^{\beta\mu}
            +e^{2\beta\mu}
        \bigg]
    \end{align}
    Density $\rho=N/V$:
    \begin{align}
        \rho&=\mp\ g_s\cdot k_BT\cdot\beta\bigg(
            \frac{\pi m}{\beta h^2}
        \bigg)^{3/2}\cdot\bigg[
            \mp\sqrt{8}e^{\beta\mu}
            +e^{2\beta\mu}
        \bigg]
    \end{align}
    With $u=\rho\lambda^3/g_s$ and $z=e^{\beta\mu}$:
    \begin{align}
        u
        &=\mp\ k_BT\cdot\beta\bigg(
            \frac{\pi m\lambda^2}{\beta h^2}
        \bigg)^{3/2}\cdot\bigg[
            \mp\sqrt{8}e^{\beta\mu}
            +e^{2\beta\mu}
        \bigg] \\
        &=\mp\ k_BT\cdot\beta\bigg(
            \frac{\pi m\lambda^2}{\beta h^2}
        \bigg)^{3/2}\cdot\bigg[
            \mp\sqrt{8}z
            +z^2
        \bigg]
    \end{align}

\paragraph{3. Combine your results for $\Psi$ and $u$ to obtain 
    the first two terms for pressure p in an expansion in 
    $\rho$ to second order. \textit{(2.5 points)}
} \ \\
    \begin{align}
        p
        &=-\Psi/V \\
        &=\mp\ g_s\cdot k_BT\cdot\bigg(
            \frac{\pi m}{\beta h^2}
        \bigg)^{3/2}\cdot\bigg[
            \mp\sqrt{8}e^{\beta\mu}
            +\frac{1}{2}e^{2\beta\mu}
        \bigg] \\
        &=...?
    \end{align}
    % \begin{align}
    %     p=-\Psi\rho/N
    % \end{align}
    \\
    \begin{align}
    pV &= - \Psi = g_s k_B T \frac{V}{\lambda^3} \left(
    z + \frac{\sigma z^2}{2^{5/2}} + ... \right) \notag \\
    &= N k_B T \left( 1-\frac{\sigma u}{2^{5/2}} + ... \right) \,.
    \end{align}
    Something like this...

\paragraph{4. Discuss your results for the classical limit and 
    the first quantum correction. Where does degeneracy $g_s$
    show up? What do the differences in sign mean?
    \textit{(1 point)}
} \ \\
    \\
    The grandcanonical potential with the minus sign 
    $\Psi_-$ describes the ideal bosonic gas, while 
    $\Psi_+$ describes the ideal fermionic gas.
    $\sigma=\pm 1$ is the difference between fermions and bosons.
    When quantum corrections are taken into account, the pressure 
    increases for fermions (Pauli exclusion principle) and decreases
    for bosons.
