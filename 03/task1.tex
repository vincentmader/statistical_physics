Consider the probability density function
\begin{equation}
    f(x)=c\bigg[\frac{\gamma}{(x-x_0)^2+\gamma^2}\bigg]
\end{equation}
with $\gamma>0$ and $-\infty<x_0<\infty$, which is known as the 
Cauchy distribution in mathematics and as the Lorentz distribution
in physics.

\paragraph{a) Calculate the normalization constant $c>0$ and the median
    of the distribution for $\gamma=1$ and $x_0=0$.
} \ \\
\\
    Any probability density function $f(x)$ must fulfill the normalization 
    condition
    \begin{align}
        1
        &=\int_{-\infty}^\infty f(x)\cdot dx \\
        &=\int_{-\infty}^\infty c\cdot\bigg[
            \frac{\gamma}{(x-x_0)^2+\gamma^2}
        \bigg]\cdot dx \\
        &=\frac{c}{\gamma}\cdot\int_{-\infty}^\infty \bigg[
            \frac{1}{(\frac{x-x_0}{\gamma})^2+1}
        \bigg]\cdot dx \\
        &=\frac{c}{\gamma}\cdot\int_{-\infty}^\infty \bigg[
            \frac{1}{z^2+1}
        \bigg]\cdot dz\ \ \ \ \ \ \ \ \ 
        \textnormal{with } z:=\frac{x-x_0}{\gamma} \\
        &=\frac{c}{\gamma}\cdot\bigg[\arctan(z)\bigg]^{\infty}_{-\infty} \\
        &=\frac{c}{\gamma}\cdot(\frac{\pi}{2}-(-\frac{\pi}{2})) \\
        &=\frac{c}{\gamma}\cdot\pi \\
        \Rightarrow c&=\frac{\gamma}{\pi}
    \end{align}
    Addition: The integral 
    \begin{align}
    \int_{\infty}^{\infty} \frac{dx}{1+x^2}
    \end{align}
    can be solved beautifully with complex analysis. Notice that from the residue theorem follows
    \begin{align}
    \int_{\infty}^{\infty} \frac{dx}{1+x^2} = \oint \frac{dx}{(x+i)(x-i)} = 2\pi i \text{Res}\left( \frac{1}{1+x^2} , i \right) = \pi.
    \end{align}
    The residue is given by
    \begin{align}
    \text{Res}\left( \frac{1}{1+x^2} , i \right) = \lim_{x \rightarrow i} \ (x - i) \frac{1}{1+x^2} = \lim_{x \rightarrow i} \ \frac{1}{x+i} = \frac{1}{2i}.
    \end{align}
    For the median $x_m$, we have to solve
    \begin{align}
        \frac{1}{2}
        &=\int_{-\infty}^{x_m}f(x)\cdot dx \\
        &=\int_{-\infty}^{x_m}c\cdot\bigg[
            \frac{\gamma}{(x-x_0)^2+\gamma^2}
        \bigg]\cdot dx \\
        &=\frac{c}{\gamma}\cdot\bigg[\arctan(z)\bigg]^{x_m}_{-\infty} \\
        &=\frac{c}{\gamma}\cdot\bigg(\arctan(x_m)+\frac{\pi}{2}\bigg) \\
        \Rightarrow x_m
        &=\tan\bigg({\frac{\gamma}{2c}-\frac{\pi}{2}}\bigg) ???
    \end{align}
    Addition: In general, every symmetric distribution has median 0 from symmetry reasons.


\paragraph{b) Try to evaluate the first two moments of the distribution.
    What problem arises? Compare the situation to the Gaussian 
    distribution and comment on the implications for large derivations
    from the median.
} \ \\
\\
    First moment:
    \begin{align}
        \int_{-\infty}^\infty x\cdot f(x)\cdot dx
        &=\int_{-\infty}^\infty c\cdot\bigg[
            \frac{\gamma\cdot x}{(x-x_0)^2+\gamma^2}
        \bigg]\cdot dx \\
        &=\frac{c}{\gamma}\cdot\int_{-\infty}^\infty\bigg[
            \frac{x}{(\frac{x-x_0}{\gamma})^2+1}
        \bigg]\cdot dx
    \end{align}
    Second moment:
    \begin{align}
        \int_{-\infty}^\infty x^2\cdot f(x)\cdot dx
        &=\int_{-\infty}^\infty c\cdot\bigg[
            \frac{\gamma\cdot x^2}{(x-x_0)^2+\gamma^2}
        \bigg]\cdot dx \\
        &=\frac{c}{\gamma}\cdot\int_{-\infty}^\infty\bigg[
            \frac{x^2}{(\frac{x-x_0}{\gamma})^2+1}
        \bigg]\cdot dx
    \end{align}
    The problem that arises is that both of these integrals do not converge. \\
    \\
    \textcolor{red}{Implications?}

