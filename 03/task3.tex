\paragraph{a) Reservoirs for energy and particle number:
    Consider a system in a fixed volume that not only exchanges energy
    but also particles with the surrounding. To calculate the 
    probability distribution $p_i$, maximize the information 
    (Shannon) entropy $S=-\sum_ip_i\log{p_i}$ under the joint 
    conditions of normalization, of $U=\sum_ip_iE_i$ (mean energy) and 
    $N=\sum_ip_iN_i$ (mean particle number). What physical meaning has 
    the new Lagrangian multiplier associated to the condition in $N$?
} \ \\
\\

\paragraph{b) Rational probabilities: if one does not know the 
    probabilities of events, one can define so-called 
    ratioanl probabilities $\bar{p}$ such that entropy is maximized 
    subject to the constraints imposed by the available information.
    Assume that in a certain game, a player can score any integer 
    $n=0,1...$ and it is known that the mean score is $\mu$. Use again
    the entropy and the method of Lagrange multipliers to show that 
    when imposing the relevant constraints the rational choice is 
    $\bar{p}_n=\frac{\mu^n}{(1+\mu)^{n+1}}$
} \ \\
\\

