\paragraph{a) Reservoirs for energy and particle number:
    Consider a system in a fixed volume that not only exchanges energy
    but also particles with the surrounding. To calculate the 
    probability distribution $p_i$, maximize the information 
    (Shannon) entropy $S=-\sum_ip_i\log{p_i}$ under the joint 
    conditions of normalization, of $U=\sum_ip_iE_i$ (mean energy) and 
    $N=\sum_ip_iN_i$ (mean particle number). What physical meaning has 
    the new Lagrangian multiplier associated to the condition in $N$?
} \ \\
\\

Consider the function 
\begin{align}
f(p_i) = - \sum_i p_i \ln p_i - \lambda_1 \left( \sum_i p_i E_i - U \right) - \lambda_2 \left( \sum_i p_i N_i - N \right).
\end{align}
The variation with respect to the probability distribution yields
\begin{align}
\delta f(p_i) = - \sum_i p_i \left( \ln p_i + 1 + \lambda_1 E_i + \lambda_2 N_i \right) \delta p_i =^{!} 0.
\end{align}
Which in fact yields:
\begin{align}
p_i = e^{- (1 + \lambda_1 E_i + \lambda_2 N_i)}.
\end{align}


\paragraph{b) Rational probabilities: if one does not know the 
    probabilities of events, one can define so-called 
    rational probabilities $\bar{p}$ such that entropy is maximized 
    subject to the constraints imposed by the available information.
    Assume that in a certain game, a player can score any integer 
    $n=0,1...$ and it is known that the mean score is $\mu$. Use again
    the entropy and the method of Lagrange multipliers to show that 
    when imposing the relevant constraints the rational choice is 
    $\bar{p}_n=\frac{\mu^n}{(1+\mu)^{n+1}}$
} \ \\
\\

Now the function would look like this:
\begin{align}
f(p_i) = - \sum_i p_i \ln p_i - \lambda_1 \left( \sum_n p_i n - \mu \right).
\end{align}
Variation with respect to $p_n$ yields
\begin{align}
p_n = e^{- (1 + \lambda_1 n)}.
\end{align}
From the normalization condition, we get:
\begin{align}
\sum_{n=0}^{\infty} p_n = 1 \ \ \ \longrightarrow \ \ \ \lambda_1 = \ln\left(\frac{e}{e-1}\right).
\end{align}
