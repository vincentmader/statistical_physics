In the lecture, the Fokker-Planck equation was derived for constant 
drift velocity $v$ and constant diffusion constant $D$. In the case 
that they are not constant, the Fokker-Planck equation reads
\begin{equation}
    \dot{p}(x,t)=-\partial_x\bigg(v(x)p(x,t)\bigg)+
    \partial_x^2\bigg(D(x)p(x,t)\bigg)
    \label{eq:fokker}
\end{equation}
To find the stationary limit, the left hand side $\dot p(x,t)$ is set 
to zero.

\paragraph{a) Consider an overdamped particle in one dimension with a 
    harmonic potential $U(x)=\frac{1}{2}kx^2$ ($k$ is the spring 
    constant). This could be e.g. a colloid in an optical trap. Use the 
    balance between friction force $\xi v$ ($\xi$ is the friction
    coefficient) and potential force $-\partial_xU(x)$ to replace 
    $v(x)$ in this equation. $D$ is assumed to be constant. Solve for 
    the stationary distribution $p_s(x)$. Note that both $p_s(x)$ and 
    its derivative should vanish at $x=\pm\infty$.
} \ \\
\\
    Balance between fiction and potential force:
    \begin{align}
        \xi v(x)
        &=-\partial_x U(x) \\
        \Rightarrow v(x)&=-\frac{1}{\xi}\partial_x U
    \end{align}
    Plug into \autoref{eq:fokker} and set $\dot p(x,t)=0$ to find 
    stationary solution:
    \begin{align}
        \partial_x\bigg(-\frac{1}{\xi}\partial_x U\cdot p(x,t)\bigg)
        &=\partial_x^2\bigg(D(x)\cdot p(x,t)\bigg) \\
        &=...
    \end{align}

\paragraph{b) Compare to the Boltzmann distribution 
    $p_s(x)\sim\exp\bigg(-\frac{U(x)}{k_BT}\bigg)$ and from this derive a 
    relation between diffusion constant $D$ and friction coefficient $\xi$.
} \ \\
\\

\paragraph{c) Calculate the first and second moments of the stationary
    distribution. Interpret your results in terms of physics.
} \ \\
\\

