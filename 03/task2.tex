In the lecture, the Fokker-Planck equation was derived for constant 
drift velocity $v$ and constant diffusion constant $D$. In the case 
that they are not constant, the Fokker-Planck equation reads
\begin{equation}
    \dot{p}(x,t)=-\partial_x\bigg(v(x)p(x,t)\bigg)+
    \partial_x^2\bigg(D(x)p(x,t)\bigg)
    \label{eq:fokker}
\end{equation}
To find the stationary limit, the left hand side $\dot p(x,t)$ is set 
to zero.

\paragraph{a) Consider an overdamped particle in one dimension with a 
    harmonic potential $U(x)=\frac{1}{2}kx^2$ ($k$ is the spring 
    constant). This could be e.g. a colloid in an optical trap. Use the 
    balance between friction force $\xi v$ ($\xi$ is the friction
    coefficient) and potential force $-\partial_xU(x)$ to replace 
    $v(x)$ in this equation. $D$ is assumed to be constant. Solve for 
    the stationary distribution $p_s(x)$. Note that both $p_s(x)$ and 
    its derivative should vanish at $x=\pm\infty$.
} \ \\
\\
    % Balance between fiction and potential force:
    % \begin{align}
    %     \xi v(x)
    %     &=-\partial_x U(x) \\
    %     \Rightarrow v(x)&=-\frac{1}{\xi}\partial_x U
    % \end{align}
    % Plug into \autoref{eq:fokker} and set $\dot p(x,t)=0$ to find 
    % stationary solution:
    % \begin{align}
    %     \partial_x\bigg(-\frac{1}{\xi}\partial_x U\cdot p(x,t)\bigg)
    %     &=\partial_x^2\bigg(D(x)\cdot p(x,t)\bigg) \\
    %     &=...
    % \end{align}
    From the balance between friction force $\xi v$ and potential force 
    $-\partial_xU(x)$ we can deduce
    \begin{align}
        \xi v(x) 
        &= -\partial_xU(x) \notag \\
        &= -kx \notag \\
        \Rightarrow v(x) 
        &= -\frac{kx}{\xi}.
    \end{align}
    Assuming $D$ to be constant, we get
    \begin{align}
    	\dot{p}(x,t)&=-\partial_x\bigg(v(x)p(x,t)\bigg)+
        \partial_x^2\bigg(D(x)p(x,t)\bigg) \notag \\
        \dot{p}(x,t)&=\frac{k}{\xi} p(x,t) + \frac{kx}{\xi} \partial_x p(x,t) +
        D \partial_x^2 p(x,t) .
    \end{align}
    The stationary solution is given where $\dot{p}(x,t) = 0$. Thus
    \begin{align}
    	0 = \frac{k}{\xi D} p(x) + \frac{kx}{\xi D} p'(x) + p''(x).
    \end{align}
    Ansatz: Gauss function $p \propto e^{-\lambda x^2}$. Then solve for $\lambda$ and find
	\begin{align}
	\lambda = \frac{k}{2D\xi}.
	\end{align}

\newpage
\paragraph{b) Compare to the Boltzmann distribution 
    $p_s(x)\sim\exp\bigg(-\frac{U(x)}{k_BT}\bigg)$ and from this derive a 
    relation between diffusion constant $D$ and friction coefficient $\xi$.
} \ \\
\\
    We see that we get $p(x)\sim\exp\bigg(-\frac{k x^2}{2D\xi}\bigg)$. In comparison to the Boltzmann
    distribution, one can see:
    \begin{align}
    \frac{U(x)}{k_BT} &= \frac{k x^2}{2D\xi} \,, \\
    \frac{1}{2k_BT} &= \frac{1}{2D\xi} \,, \\
    k_BT &= \xi D \,.
    \end{align}

\paragraph{c) Calculate the first and second moments of the stationary
    distribution. Interpret your results in terms of physics.
} \ \\
\\
    First, you have to find the normalization $c$ then calculate the integrals
    \begin{align}
    <x> &= c \int_{-\infty}^{\infty} dx \ x e^{-\lambda x^2} \,, \\
    <x^2> &= c \int_{-\infty}^{\infty} dx \ x^2 e^{-\lambda x^2} \,.
    \end{align}
	From symmetry, one can see that the first moment is zero because we integrate an anti-symmetric function over a symmetric intervall. \\
	Carrying out the other integral yields for the second moment: $<x^2> = D\xi / k$.