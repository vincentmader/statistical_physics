In the lecture, the Fokker-Planck equation was derived for constant 
drift velocity $v$ and constant diffusion constant $D$. In the case 
that they are not constant, the Fokker-Planck equation reads
\begin{equation}
    \dot{p}(x,t)=-\partial_x\bigg(v(x)p(x,t)\bigg)+
    \partial_x^2\bigg(D(x)p(x,t)\bigg)
\end{equation}
To find the stationary limit, the left hand side $\dot p(x,t)$ is set 
to zero.

\paragraph{a) Consider an overdamped particle in one dimension with a 
    harmonic potential $U(x)=\frac{1}{2}kx^2$ ($k$ is the spring 
    constant). This could be e.g. a colloid in an optical trap. Use the 
    balance between friction force $\xi v$ ($\xi$ is the friction
    coefficient) and potential force $-\partial_xU(x)$ to replace 
    $v(x)$ in this equation. $D$ is assumed to be constant. Solve for 
    the stationary distribution $p_s(x)$. Note that both $p_s(x)$ and 
    its derivative should vanish at $x=\pm\infty$.
} \ \\
\\

\paragraph{b) Compare to the Boltzmann distribution 
    $p_s(x)\sim\exp(-\frac{U(x)}{k_BT}$ and from this derive a relation
    between diffusion constant $D$ and friction coefficient $\xi$.
} \ \\
\\

\paragraph{Calculate the first and second moments of the stationary
    distribution. Interpret your results in terms of physics.
} \ \\
\\

