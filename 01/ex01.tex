\documentclass[11 pt]{article}

% \usepackage{equation}
\usepackage[left=2cm,right=2cm,top=2cm,bottom=2cm]{geometry}
\usepackage{fancyhdr}

\pagestyle{fancy}
\fancyhf{}
\chead{Exercise 01}
\lhead{Theoretical Statistical Physics}
\rhead{E. Dizer, V. Mader}
\setlength{\headheight}{15pt}


\begin{document}

    \section{Gambling in the lottery}
        In the lottery, six numbered balls are drawn randomly from a sample of 
        49 balls carrying the numbers 1 to 49. The balls are drawn 
        sequentially and are not placed back in the sample, so that each number
        can only occur at most once. In every drawing step, each of the 
        remaining balls is chosen with equal probability. To win the lottery,
        you have to guess the six random numbers, where the order of the 
        numbers is irrelevant.

        \subsection{Calculate the number of possible sequences of 6 out of 49}

            \begin{equation}
                N
                =49\cdot48\cdot47\cdot46\cdot45\cdot44
                =10068347520
            \end{equation}

        \subsection{Calculate the probability that you guess all six numbers correctly}

            \begin{equation}
                P
                =\frac{1}{N}
                % &=?
            \end{equation}

        \subsection{Calculate the probability that you guess exactly four numbers correctly}

            ...

    \section{Birthdays}
        Assuming that the birthdays of a population are distributed equally 
        over the 365 days of a year, how large is the probability that in a 
        lecture attended by 150 students at least two of the students have 
        their birthday on the same day of the year? How many people would have
        to attend the lecture so that the probability for them having their 
        birthday on the same day is 0.5?

    \section{Bayes and a-posteriori probabilities}
        Bayes theorem can be used to improve/revise our guesses on 
        probabilities.


\end{document}
