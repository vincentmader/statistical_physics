A metal contains conduction electrons in a volume $V$. 
The temperature is $T=0$ so that the chemical potential is 
$\mu=\varepsilon_F$. The magnetic moment of the electrons 
is $\mu_m$. A magnetic field B is switched on, hence the 
energy of particles with spin parallel (+) and anti-parallel 
(-) to the magnetic field is
\begin{equation}
    \varepsilon_\pm(p)=\frac{p^2}{2m}\mp\mu_mB
\end{equation}
respectively. In the following the magnetic field is assumed 
so weak that only its first order effect should be considered.

\paragraph{a) Calculate the mean number $\langle N_\pm\rangle$ 
    of electrons with spin parallel and anti-parallel to the 
    magnetic field, respectively. \textit{(3 points)}
} \ \\
    \\
    \begin{equation}
        Z=Z_++Z_- \, ???
    \end{equation}
    \begin{equation}
        Z_\pm=
        \int_0^{\infty}\dd\varepsilon\cdot\bigg(
            1+\exp(-\beta\varepsilon)
        \bigg)\cdot n_\pm(\varepsilon)
    \end{equation}

    % \begin{equation}
    %     N=
    %     \int d\varepsilon\cdot D(\varepsilon) 
    %     \cdot n_+(\varepsilon)\cdot n_-(\varepsilon)
    % \end{equation}

    \begin{equation}
        \mean{N}
        =-\frac{1}{\beta}\partial_\mu\ln(Z)
    \end{equation}


\paragraph{b) Calculate the mean magnetization $M$ and the magnetic 
    susceptibility $\chi$. \textit{(2 points)}
} \ \\
    \\
    \begin{equation}
        M=\mu_m\cdot(\langle N_+\rangle - \langle N_-\rangle)
    \end{equation}
    \begin{equation}
        \chi=\bigg(\pder{M}{B}\bigg)
    \end{equation}


\paragraph{c) Express the chemical potential $\mu$ in terms of the 
    mean total number of electrons $\langle N\rangle$ and 
    eliminate the chemical potential from the formula 
    for the susceptibility. \textit{(2 points)}
}
    \begin{equation}
        \langle N\rangle
        =\langle N_+\rangle+\langle N_-\rangle
    \end{equation}
