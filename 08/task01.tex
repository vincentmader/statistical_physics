A metal contains conduction electrons in a volume $V$. 
The temperature is $T=0$ so that the chemical potential is 
$\mu=\varepsilon_F$. The magnetic moment of the electrons 
is $\mu_m$. A magnetic field B is switched on, hence the 
energy of particles with spin parallel (+) and anti-parallel 
(-) to the magnetic field is
\begin{equation}
    \varepsilon_\pm(p)=\frac{p^2}{2m}\mp\mu_mB
\end{equation}
respectively. In the following the magnetic field is assumed 
so weak that only its first order effect should be considered.

\paragraph{a) Calculate the mean number $\langle N_\pm\rangle$ 
    of electrons with spin parallel and anti-parallel to the 
    magnetic field, respectively. \textit{(3 points)}
} \ \\
    \\
    % \begin{equation}
    %     Z=Z_++Z_-
    % \end{equation}
    % \begin{equation}
    %     Z_\pm=
    %     \int_0^{\infty}\dd\varepsilon\cdot\bigg(
    %         1+\exp(-\beta\varepsilon)
    %         \ \red{?}
        % \bigg)\cdot n_\pm(\varepsilon)
    % \end{equation}
    Total number of particles:
    \begin{align}
        \langle N\rangle
        &=2\cdot\frac{V}{h^3}\cdot\int_0^{p_F}d^3p \\
        &=2\cdot\frac{V}{h^3}\cdot4\pi\cdot\int_0^{p_F}p^2dp \\
        &=2\cdot\frac{4\pi V}{3h^3}\cdot p_F^3 \\
        \Rightarrow \mean{N_\pm}
        &=\frac{4\pi V}{3h^3}p^3_{F\pm}
    \end{align}
    with
    \begin{align}
        p_{F+}=\sqrt{
            2m(\mu-\mu_mB)
        }
        \ \ \ \ \textnormal{and}\ \ \ \
        p_{F-}=\sqrt{
            2m(\mu+\mu_mB)
        }
    \end{align}
    % \begin{equation}
    %     Z=Z_++Z_- \, ???
    % \end{equation}
    % \begin{equation}
    %     Z_\pm=
    %     \int_0^{\infty}\dd\varepsilon\cdot\bigg(
    %         1+\exp(-\beta\varepsilon)
    %     \bigg)\cdot n_\pm(\varepsilon)
    % \end{equation}
    % \begin{equation}
    %     N=
    %     \int d\varepsilon\cdot D(\varepsilon) 
    %     \cdot n_+(\varepsilon)\cdot n_-(\varepsilon)
    % \end{equation}
    % \begin{equation}
    %     \mean{N_\pm}
    %     =-\frac{1}{\beta}\partial_\mu\ln(Z_\pm) 
    %     =\frac{4\pi V}{3h^3}p^3_{F\pm}
    % \end{equation}


\paragraph{b) Calculate the mean magnetization $M$ and the magnetic 
    susceptibility $\chi$. \textit{(2 points)}
} \ \\
    % \\
    \begin{align}
        M
        &=\mu_m\cdot(\langle N_+\rangle-\langle N_-\rangle) \\
        &=\mu_B\cdot\bigg(
            \frac{4\pi V}{3h^3}p_{F+}^3-
            \frac{4\pi V}{3h^3}p_{F-}^3
        \bigg) \\
        &=\mu_B\cdot\frac{4\pi V}{3h^3}\cdot\bigg(
            [2m\cdot(\mu-\mu_mB)]^{3/2}-
            [2m\cdot(\mu+\mu_mB)]^{3/2}
        \bigg) \\
    % \end{align}
    % \begin{align}
        \Rightarrow\chi
        &=\bigg(\pder{M}{B}\bigg) \\
        &=\mu_m\cdot\partial_B\bigg(
            \langle N_+\rangle-\langle N_-\rangle
        \bigg) \\
        &=\mu_m\cdot\bigg(
            \frac{3}{2}\cdot\sqrt{2m\cdot(\mu+\mu_mB)}
            -\frac{3}{2}\cdot\sqrt{2m\cdot(\mu-\mu_mB)}
        \bigg) \\
        &=3\sqrt{\frac{m}{2}}\cdot\mu_m\cdot\bigg(
            \sqrt{\mu+\mu_mB}-\sqrt{\mu-\mu_mB}
        \bigg) 
    \end{align}


\paragraph{c) Express the chemical potential $\mu$ in terms of the 
    mean total number of electrons $\langle N\rangle$ and 
    eliminate the chemical potential from the formula 
    for the susceptibility. \textit{(2 points)}
}
    \begin{equation}
        \langle N\rangle
        =\langle N_+\rangle+\langle N_-\rangle
    \end{equation}
