An ultra-relativistic ideal Fermi fluid is contained in a 
volume $V$. The chemical potential is $\mu$. Consider only 
the special case $T=0$ in the following. Calculate the mean 
particle number $\langle N \rangle$ and the mean total energy
$\langle E \rangle$.
Express $\langle E \rangle$ in terms of $\langle N \rangle$ and 
$\mu$. \\
\\
HINT: Ultra-relativistic means that you can neglect the mass 
term in the relativistic formula for the energy of the 
particles:
\begin{equation}
    \varepsilon(\vec p)=\sqrt{m^2c^4+\vec p^2c^2}
    \approx pc \,.
\end{equation} \\
\\
We use the following identity from the lecture script:
\begin{align}
    \sum_{\vec k,m_S} (...)
    =2\cdot\frac{V}{h^3}\int\dd\vec p \, (...)
    =\frac{2V}{h^3}\int_0^{\infty}\dd p \, 4 \pi p^2 (...) \,.
\end{align}
Thus, the mean particle number is given by
\begin{align}
    \langle N \rangle = \sum_{\vec k,m_S} n_{\vec k,m_S}
    =\frac{8\pi V}{c^3h^3}\int_0^{\infty}\dd\varepsilon \, \varepsilon^2 n(\varepsilon) \,,
\end{align}
where we used the linear dispersion relation $\varepsilon(p)=pc$ for 
ultra-relativistic particles. In the limit $T=0$, the Fermi function
becomes $n(\varepsilon) = \theta(\mu - \varepsilon)$:
\begin{align}
    \langle N \rangle
    =\frac{8\pi V}{c^3h^3}\int_0^{\mu}\dd\varepsilon \, \varepsilon^2 
    =\frac{8\pi V}{3c^3h^3} \mu^3 \,.
\end{align}
A similar calculation is done for the mean energy:
\begin{align}
    \langle E \rangle
    =\frac{8\pi V}{c^3h^3}\int_0^{\infty}\dd\varepsilon \, \varepsilon^2 n(\varepsilon)\varepsilon
    =\frac{8\pi V}{c^3h^3}\int_0^{\mu}\dd\varepsilon \, \varepsilon^3
    =\frac{2\pi V}{c^3h^3}\mu^4
    =\frac{3}{4} \langle N \rangle \mu \,.
\end{align}