An ultra-relativistic ideal Fermi fluid is contained in a 
volume $V$. The chemical potential is $\mu$. Consider only 
the special case $T=0$ in the following. Calculate the mean 
particle number ⟨N⟩ and the mean total energy ⟨E⟩. 
Express ⟨E⟩ in terms of ⟨N⟩ and µ. \\
\\
HINT: Ultra-relativistic means that you can neglect the mass 
term in the relativistic formula for the energy of the 
particles:
\begin{equation}
    \varepsilon(\vec p)=\sqrt{m^2c^4+\vec p^2c^2}
    \approx |\vec p|c
    =\hbar c\cdot|\vec k|
\end{equation}
From the lecture script:
\begin{align}
    N
    &=\sum_{\vec k,m_S\textnormal{  for } p\le p_F} \\
    &=2\frac{V}{h^3}\int_{p\le p_F}\dd\vec p \\
    &=\frac{2V}{h^3}\cdot\frac{4\pi}{3}\cdot p_F^3 \\
    &=\frac{2V}{h^3}\cdot
    \frac{4\pi}{3}\cdot\bigg(\frac{E_F}{c}\bigg)^3 \\
    &=\frac{2V}{h^3}\cdot
    \frac{4\pi}{3}\cdot\bigg(\frac{\mu}{c}\bigg)^3
\end{align}
\red{?}
$\langle E\rangle=\frac{3}{5}=E_F\frac{3}{5}\mu$
% Partition sum for fermions:
% \begin{equation}
%     Z_G=\prod_{\vec k,m_S}z_{\vec k,m_S}
% \end{equation}
% with 
% \begin{equation}
%     z_{\vec k,m_S}=1+\exp\bigg(
%         -\beta\cdot(\varepsilon_{\vec k,m_S}-\mu)
%     \bigg)
% \end{equation}
% and 
% \begin{equation}
%     \varepsilon_{\vec k,m_S}
%     \approx\hbar c\cdot|\vec k|
% \end{equation}
% Mean energy $\mean{E}$:
% \begin{equation}
%     \mean{E}
%     =-\partial_\beta\ln(Z)
% \end{equation}
% Mean particle number $\mean{N}$:
% \begin{equation}
%     \mean{N}
%     =-\frac{1}{\beta}\partial_\mu\ln(Z)
% \end{equation}
