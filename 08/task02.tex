Consider a system of spin $1/2$ fermions at temperature $T$, 
that occupy a finite number $M$ of energy levels 
$\varepsilon_i$ with $i=1,2,...,M$.


\paragraph{a) Determine the grand canonical partition sum $Z_G$.
    \textit{(1 point)}
} \ \\
    \\
    HINT: Pay attention to the degeneracy. \\
\\
We can have the following configurations for one energy level $\varepsilon_i$: 
\begin{itemize}
	\item[1.] The energy level is not occupied ($E=0$).
	\item[2.] The energy level is occupied with one spin-up electron ($E=\varepsilon_i$).
	\item[3.] The energy level is occupied with one spin-down electron ($E=\varepsilon_i$).
	\item[4.] The energy level is occupied with two electron ($E=2\varepsilon_i$).
\end{itemize}
Since spin-up and spin-down electrons are distinguishable, we have two microstates with 
energy $\varepsilon_i$. Thus, the partition sum for one energy level is:
\begin{align}
	z_i = 1+2e^{-\beta(\varepsilon_i-\mu)}+e^{-2\beta(\varepsilon_i-\mu)}
	= \left( 1+e^{-\beta(\varepsilon_i-\mu)} \right)^2 \,.
\end{align}
Remark: In general, for degeneracy $g$, the partition sum would be given by
\begin{align}
	z_i = \left( 1+e^{-\beta(\varepsilon_i-\mu)} \right)^g \,.
\end{align}
For $M$ independent energy levels, the grand canonical partition sum is
\begin{align}
	Z_G = \prod_{i=1}^M \left( 1+e^{-\beta(\varepsilon_i-\mu)} \right)^2 \,.
\end{align}

\paragraph{b) Use the relation $F=-k_BT\ln Z_G+\mu\langle N\rangle$
    between the free energy, the grand canonical partition 
    sum, the chemical potential and the mean particle number 
    to show: the thermodynamic properties of the system stay 
    the same if one, instead of considering the 
    ⟨N⟩ particles, distributes $2M-\langle N\rangle$ 
    "holes" with chemical potential $-\mu$ on the energy levels
    $-\varepsilon_i$.
    \textit{(3 points)}
} \ \\
\\
You need to show that the free energy (from which all the physical properties 
of the system are determined by derivatives) of the "holes" differs at most by
a constant term: $F' = F + C$.
The free energy of the particles is
\begin{align}
	F = \mu\langle N\rangle - 2k_BT \sum_{i=1}^M \ln\left( 1+e^{-\beta(\varepsilon_i-\mu)} \right) \,.
\end{align}
The free energy of the "holes" is
\begin{align}
	F' = -\mu(2M-\langle N\rangle) - 2k_BT \sum_{i=1}^M \ln\left( 1+e^{\beta(\varepsilon_i-\mu)} \right) \,.
\end{align}
Indeed one can show that
\begin{align}
	F' = F - 2\sum_{i=1}^M \varepsilon_i \,.
\end{align}