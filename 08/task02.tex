Consider a system of spin $1/2$ fermions at temperature $T$, 
that occupy a finite number $M$ of energy levels 
$\varepsilon_i$ with $i=1,2,...,M$.


\paragraph{a) Determine the grand canonical partition sum $Z_G$.
    \textit{(1 point)}
} \ \\
    \\
    HINT: Pay attention to the degeneracy.
    \begin{equation}
        Z_G=
        \sum_{i=1}^M\bigg[
            1+2e^{
                -\beta(\varepsilon_i-2\mu)  % 2 = Zahl der Teilchen fuer E_i
            }+e^{
                -\beta(\varepsilon_i-\mu)
            }
        \bigg]
    \end{equation}

\paragraph{b) Use the relation $F=-k_BT\ln Z_G+\mu\langle N\rangle$
    between the free energy, the grand canonical partition 
    sum, the chemical potential and the mean particle number 
    to show: the thermodynamic properties of the system stay 
    the same if one, instead of considering the 
    ⟨N⟩ particles, distributes $2M-\langle N\rangle$ 
    "holes" with chemical potential $\mu$ on the energy levels
    $-\varepsilon_i$.
}
    % \begin{equation}
    %     F=-k_BT\ln\bigg[
    %         \sum_1^M2\cdot\exp\bigg(
    %             -\beta (\varepsilon_i-2\mu)  % 2 = Zahl der Teilchen fuer E_i
    %         \bigg)+\exp\bigg(
    %             -\beta(\varepsilon_i-\mu)
    %         \bigg)
    %     \bigg]+\mu\langle N\rangle
    % \end{equation}
    % \begin{equation}
    %     Z_G=
    %     \sum_1^M2\cdot\exp\bigg(
    %         -\beta (-\varepsilon_i-2\mu)  % 2 = Zahl der Teilchen fuer E_i
    %     \bigg)
    % \end{equation}
