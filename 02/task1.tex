Start with the definition of the continuous factorial:
\begin{align}
    n!
    &=\int_0^\infty x^n\cdot e^{-x}\cdot dx \\
    &=\int_0^\infty e^{-x+n\log x}\cdot dx \\
    &=n\cdot\int_0^\infty e^{-ny+n\log(ny)}\cdot dy
\end{align}
With $f(y)=-y+\log(ny)$ this can be written as
\begin{align}
    &=n\cdot\int_0^\infty e^{nf(y)}\cdot dy.
\end{align}
The maximum of $f(y)$ is at $y_0=1$. Taylor expansion 
around $y_0$:
\begin{equation}
    f(y)\approx\log(n)-1-\frac{(y-1)^2}{2}.
\end{equation}
Thus
\begin{align}
	n!
    &\approx n e^{n\log(n) -n} \cdot\int_0^\infty e^{-n(y-1)^2/2}\cdot dy \notag \\
    &= n e^{n\log(n) -n} \sqrt{\frac{\pi}{2n}}\left(1 + \text{Erf}\left(\sqrt{\frac{n}{2}}\right)\right) \notag \\
    &\approx  \sqrt{2 \pi n} e^{n\log(n) -n} ,
\end{align}
where we have used Erf$(x) \approx 1$ for large $x$. Taking the logarithm yields the Stirling formula:
\begin{align}
    \log(n!) = n \log(n) - n + \frac{1}{2} \log(2\pi n).
\end{align}