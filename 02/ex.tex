\documentclass[11 pt]{article}

\usepackage{amsmath}
\usepackage[left=2cm,right=2cm,top=2cm,bottom=2cm]{geometry}
\usepackage{fancyhdr}
\usepackage[hidelinks]{hyperref}

\pagestyle{fancy}
\fancyhf{}
\lhead{Theoretical Statistical Physics}
\chead{Exercise 02}
\rhead{E. Dizer, V. Mader}
\setlength{\headheight}{15pt}


\begin{document}

    \section{Stirling's formula}

        \begin{align}
            n!
            &=\int_0^\infty x^n\cdot e^{-x}\cdot dx \\
            &=\int_0^\infty e^{-x+n\log x}\cdot dx \\
            &=n\cdot\int_0^\infty e^{-ny+n\log(ny)}\cdot dy
        \end{align}
        With $f(y)=-y+\log(ny)$ this can be written as
        \begin{align}
            &=n\cdot\int_0^\infty e^{nf(y)}\cdot dy
        \end{align}
        The maximum of $f(y)$ is at $y_0=1$. Taylor expansion 
        around $y_0$:
        \begin{equation}
            f(y)\approx\log(y)-1-y^2
        \end{equation}

    \newpage
    \section{Adding two Gaussian distributions}
        
        \paragraph{a) without characteristic function} \ \\ 
        \\
        Let $p_x$ and $p_y$ be Gaussian distributions with 
        parameters $\mu_x$, $\mu_y$ and $\sigma$.
        \begin{equation}
            p_x(x)=\frac{1}{\sqrt{2\pi\sigma}}
            \exp\bigg(-\frac{(x-\mu_x)^2}{2\sigma^2}\bigg)
        \end{equation}
        \begin{equation}
            p_y(y)=\frac{1}{\sqrt{2\pi\sigma}}
            \exp\bigg(-\frac{(y-\mu_y)^2}{2\sigma^2}\bigg)
        \end{equation}
        \begin{align}
            p_z(z)
            &=\int dx\int dy\cdot\delta(z-(x+y))\cdot p(x,y) \\
            &=\int dx\int dy\cdot\delta(z-(x+y))\cdot p(x)\cdot p(y) \\
            &=\frac{1}{2\pi\sigma^2}\int dx\int dy\cdot
            \delta(z-(x+y))\cdot
            \exp\bigg(-\frac{(y-\mu_y)^2}{2\sigma^2}\bigg)\cdot
            \exp\bigg(-\frac{(x-\mu_x)^2}{2\sigma^2}\bigg) \\
            &=
            &=\frac{1}{\sqrt{4\pi\sigma^2}}\cdot
            \exp\bigg(-\frac{(z-\mu_x-\mu_y)^2}{4\sigma^2}\bigg)
        \end{align}
        With $\mu_z=\frac{\mu_x+\mu_y}{2}$
        and $\sigma_z=\frac{\sigma}{\sqrt{2}}$, this can be 
        rewritten as
        \begin{equation}
            p_z(z)
            =\frac{1}{\sqrt{2\pi\sigma_z^2}}\cdot
            \exp\bigg(-\frac{(z/2-\mu_z)^2}{2\sigma^2}\bigg)
        \end{equation}
            % \mu_=
            % \sigma_z=
            % \footnote{
            % \url{ en.wikipedia.org/wiki/Sum_of_normally_distributed_random_variables } 
            % }

        \paragraph{b) with characteristic function} \ \\


    \section{Computer exercise on random numbers}

\end{document}
