Bethe proposed a refined mean field theory, where one considers \textit{clusters} of spins. The simplest (smallest)
cluster is the one taking the nearest neighbours of a chosen spin, $s_0$, into account. Those neighbours
interact exactly with $s_0$, but see their own further neighbours only via an effective mean field $\bar{B}$. The Hamiltonian hence reads:
\begin{align}
	\mathcal{H}_{\text{Bethe}}(s_0,\{s_i\})=-J s_0\sum_{i=1}^z s_i -\bar{B}\sum_{i=1}^z s_i \,.
\end{align}
Self-consistency again demands $\langle s_0 \rangle = \langle s_i \rangle$ for any $i$.

\paragraph{1. Calculate the partition sum and use it to determine $\langle s_0 \rangle$ and $\langle s_i \rangle$. \textit{(3 points)}
} \ \\
\\
The partition sum is given by
\begin{align}
	Z &= \sum_{s_0=\pm 1} \sum_{\{s_i\}} e^{\beta(Js_0+\bar{B})\sum_{i=1}^z s_i} \notag \\
	&= \sum_{s_0=\pm 1} \sum_{\{s_i\}} \prod_{i=1}^z e^{\beta(Js_0+\bar{B}) s_i} \notag \\
	&= \sum_{\{s_i\}} \prod_{i=1}^z \left( e^{\beta(J+\bar{B}) s_i} + e^{\beta(-J+\bar{B}) s_i} \right) \notag \\
	&= \left( 2\cosh\beta(J+\bar{B}) \right)^z + \left( 2\cosh\beta(J-\bar{B}) \right)^z \,.
\end{align}
The two mean values are given by
\begin{align}
	Z \langle s_0 \rangle &= \sum_{s_0=\pm 1} \sum_{\{s_i\}} s_0 \, e^{\beta(Js_0+\bar{B})\sum_{i=1}^z s_i} \notag \\
	&= \left( 2\cosh\beta(J+\bar{B}) \right)^z - \left( 2\cosh\beta(J-\bar{B}) \right)^z \,.
\end{align}
\begin{align}
	Z \langle s_i \rangle &= \sum_{s_0=\pm 1} \sum_{\{s_j\}} s_i \, e^{\beta(Js_0+\bar{B})\sum_{j=1}^z s_j} \notag \\
	&= \sum_{\{s_j\}} \prod_{j=1}^z \left( s_i \, e^{\beta(J+\bar{B}) s_j} + s_i \, e^{\beta(-J+\bar{B}) s_j} \right) \notag \\
	&= 2\sinh\beta(J+\bar{B})\left( 2\cosh\beta(J+\bar{B}) \right)^{z-1} \notag \\
	&- 2\sinh\beta(J-\bar{B})\left( 2\cosh\beta(J-\bar{B}) \right)^{z-1} \,.
\end{align}

\newpage

\paragraph{2. Show that the consistency condition then implies
\begin{align}
	\left(\frac{\cosh\beta(J+\bar{B})}{\cosh\beta(J-\bar{B})}\right)^{z-1}=e^{2\beta\bar{B}} \,.
\end{align}
Show that this equation always has one solution and that (two) more may exist. \textit{(3 points)} \\
\textsc{Hint}: Study the behavior for large $\bar{B}$
as well as the slopes of the left and right sides at $\bar{B}=0$.
} \ \\
\\
Equating $\langle s_0 \rangle$ and $\langle s_i \rangle$, one obtains:
\begin{align}
	\frac{\cosh\beta(J+\bar{B}) - \sinh\beta(J+\bar{B})}{\cosh\beta(J-\bar{B}) - \sinh\beta(J-\bar{B})}
	=  \left( \frac{\cosh\beta(J-\bar{B})}{\cosh\beta(J+\bar{B})} \right)^{z-1} \,.
\end{align}
Noting that $\cosh x - \sinh x = e^{-x}$, one obtains further:
\begin{align}
	\label{B}
	e^{2\beta \bar{B}}
	=  \left( \frac{\cosh\beta(J+\bar{B})}{\cosh\beta(J-\bar{B})} \right)^{z-1} \,.
\end{align}


\paragraph{3. Show that the critical temperature for the phase transition is 
$k_B T_c = \frac{2J}{\ln(\frac{z}{z-1})}$. \textit{(1 point)}
} \ \\
\\
Again, we obtain the solution of Eq.~\eqref{B} by equating the slopes of the LHS and the RHS:
\begin{align}
	\label{B}
	2\beta_c =  2 (z-1) \beta_c \tanh(\beta_c J) \, \, \, \rightarrow \, \, \,
	k_B T_c = \frac{J}{\text{arcoth}(\frac{1}{z-1})} = \frac{2J}{\ln(\frac{z}{z-2})} \,.
\end{align}


\paragraph{4. Discuss in how far this result is better than the one developed in 11.1. \textit{(1 point)}
} \ \\
\\
It is better than the result developed in 11.1 because at least for $d=1$ we get a qualitatively correct answer:
There is no phase transition at finite temperature, $T_c=0$. For $d=2$, there is a phase transition at finite 
temperature which is also correct.