Consider a one-dimensional gas of $N$ particles of length $a$ confined to a strip of length $L$. The particles
cannot overlap with each other (hard core repulsion) and otherwise do not interact with each other (no
attraction like in the van der Waals gas).

\paragraph{1. Calculate the canonical partition sum $Z$ by integrating over all possible values for the midpoints
of the gas particles. \textit{(2 points)}
} \ \\
\\
The canonical partition sum $Z$ is as usual a sum over all configurations in phase space:
\begin{align}
	Z &= \frac{1}{N!}\frac{1}{h^N} \int\dd^Np\int\dd^Nx \, e^{-\beta \left( \frac{p^2}{2m} + U(x) \right)} \notag \\
	&= \frac{1}{N!} \left( \frac{1}{h} \int\dd p \, e^{-\beta \frac{p^2}{2m}} \right)^N \int\dd^Nx \, e^{-\beta U(x)} \notag \\
	&= \frac{1}{N! \lambda^N} \int\dd x_1 \, ... \, \dd x_N \, e^{-\beta U(x)} \,.
\end{align}
Here, we write $\lambda$ for the thermal wavelength as usual and $U(x)$ for the potential of the system which depends on the midpoints 
$x_1, ..., x_N$ of the particles. We see that the partition sum reduces to an integral over all possible values for the midpoints 
of the gas particles. For hard core repulsion, we know that the weight $e^{-\beta U(x)}$ is either 0 (if any two particles overlap 
which is not allowed) or 1 (if no particles overlap). Thus, we can reduce the integral to the case where $x_1 < x_2 < ... < x_N$ 
and multiply by $N!$ to include all possibilities. Clearly, $x_i$ must then lie between $x_{i-1}+a$ and $L-(N-i)a-a/2$. With 
successive integration we get
\begin{align}
	Z &= \frac{1}{\lambda^N} \int_{a/2}^{L-(N-1)a-a/2}\dd x_1  \int_{x_1+a}^{L-(N-2)a-a/2}\dd x_2 \, \, ... \int_{x_{N-1}+a}^{L-a/2}\dd x_N \notag \\
	&= \frac{1}{\lambda^N} \int_{a/2}^{L-(N-1)a-a/2}\dd x_1  \int_{x_1+a}^{L-(N-2)a-a/2}\dd x_2 \, \, ... \int_{x_{N-2}+a}^{L-a-a/2}\dd x_{N-1} \, (L-3a/2-x_{N-1}) \notag \\
	&= \frac{1}{2\lambda^N} \int_{a/2}^{L-(N-1)a-a/2}\dd x_1  \int_{x_1+a}^{L-(N-2)a-a/2}\dd x_2 \, \, ... \int_{x_{N-3}+a}^{L-2a-a/2}\dd x_{N-2} \, (L-5a/2-x_{N-2})^2 \notag \\
	&= ... \notag \\
	&= \frac{1}{i! \lambda^N} \int_{a/2}^{L-(N-1)a-a/2}\dd x_1  \int_{x_1+a}^{L-(N-2)a-a/2}\dd x_2 \, \, ... \int_{x_{N-i-1}+a}^{L-ia-a/2}\dd x_{N-i} \, (L-(2i+1)a/2-x_{N-i})^i \notag \\
	&= ... \notag \\
	&= \frac{1}{(N-1)! \lambda^N} \int_{a/2}^{L-(N-1)a-a/2}\dd x_1 \, (L-(2N-1)a/2-x_1)^{N-1} \notag \\
	&= \frac{1}{N! \lambda^N} \, (L-Na)^N \,.
\end{align}
This makes also sense from a physical point of view: It's like we have $N$ independent particles which can move freely in a one-dimensional volume 
$(L-Na)$.


\newpage

\paragraph{2. Calculate the free energy $F=-k_BT\ln Z$ and the pressure $p=-\partial_L F$. Evaluate the virial
coefficient $B_2$ from the appropriate Mayer function and show that your result agrees with the
virial expansion based on the exact solution. \textit{(2 points)}
} \ \\
\\
The free energy is
\begin{align}
	F = - k_B T \left( N \ln\left( \frac{L-Na}{\lambda} \right) - \ln\left(N!\right) \right) \,.
\end{align}
The pressure is
\begin{align}
	p = \frac{N k_B T}{L-Na} \,.
\end{align}
The virial coefficient $B_2$ for one-dimensional hard particles with size $a$ is
\begin{align}
	B_2 = -\frac{1}{2} \int_{-\infty}^{\infty} \dd x \, \left( e^{-\beta U(x)} - 1 \right) = \frac{1}{2} \int_{-a}^{a} \dd x = a \,,
\end{align}
since $e^{-\beta U(r)}$ can be either 0 (when two balls intersect, $\vert x\vert \leq a$) or 1 (when $\vert x\vert > a$) as explained above. 
In comparison with the exact solution,
\begin{align}
	p = \frac{N k_B T}{L-Na} = \frac{n k_B T}{1-na} \approx n k_B T \left( 1+na+n^2a^2+... \right) \,,
\end{align}
we see that the virial coefficient $B_2$ reproduces the coefficient in the expansion of the exact result. 
However, this is not the case in general.